
\begin{Exercise}[label={frame_ground_acceleration}]
The portal frame of example \ref{frames_modification} (a) is subject to a sinusoidal ground vibration with horizontal acceleration amplitude of \qty{2}{m/s^2}. Assuming a damping ratio of \qty{5}{\%}, determine the maximum displacement and maximum total acceleration of the frame when the period of floor vibration is: (a) 0.1 seconds; (b) 0.9 seconds and (c) 5 seconds.

\begin{center}
\pictures{porticoForced}
\end{center}

\shortAnswer (a) $u_0 = \SI{0.00064}{cm}$, $\ddot{u}_0 = \SI{0.011}{m/s^2}$; (b) $u_0 = \SI{41}{cm}$, $\ddot{u}_0 = \SI{20}{m/s^2}$ and (c) $u_0 = \SI{4.2}{cm}$, $\ddot{u}_0 = \SI{0.066}{m/s^2}$
\end{Exercise}



\begin{Answer}[ref={frame_ground_acceleration}]
The maximum displacements and accelerations of a vibrating structure under sinusoidal excitation are obtained from the stationary or homogeneous solution to the differential equation,
\begin{align*}
m\ddot{u} + ku = F_0\sin(\Omega t + \varphi) \\
u(t) = \frac{F_0}{k}H\sin(\Omega t + \varphi - \Delta\varphi)
\end{align*}

where $H$ is known as the magnification factor and is equal to
$$
H = \frac{1}{\sqrt{(1-\gamma^2)^2 + 4\xi^2\gamma^2}} \quad ; \quad
\gamma = \frac{\Omega}{\omega} = \frac{T_{struct}}{T_{force}}
$$

From the solution to example 3 (a) we know that the period is $T=0.9$ seconds and the frequency is $\omega = 2\pi/T \approx \SI{7}{rad/s}$ and the lateral stiffness is $k=\SI{487}{KN/m}$. The damping ratio is $\xi=\SI{5}{\%}$.

The amplitude $F_0$ of the external force is computed from the amplitude of the ground acceleration and the moving mass of the structure
$$
F_0 = ma_{0} = \frac{Pa_0}{g} = \frac{100\cdot2}{10} = \qty{20}{KN}
$$

Finally, the maximum displacements will depend on the period $\Omega$ of the external force,
\begin{align*}
&u_0 = \frac{F_0}{k}H = \frac{20}{487}H = 0.041H\,\si{m} = 4.1H\,\si{cm} \\
&\ddot{u}_0 = \frac{F_0}{m}\gamma^2H = \Omega^2u_0
\end{align*}


When the period of the ground acceleration is $T_g=\SI{0.15}{s}$, the maximum displacements and accelerations are
\begin{align*}
&\gamma = \frac{0.9}{0.15} = 9 \quad ; \quad
H = \frac{1}{\sqrt{(1-9^2)^2 + 4\cdot 0.05^2\cdot 9^2}} = \frac{1}{\sqrt{80^4 + \cdots}} = 0.00015 \\
&u_0 = 4.1\cdot 0.00015 = \SI{0.00064}{cm} \\
&\ddot{u}_0 = \left(\frac{2\pi}{0.15}\right)^2 \cdot 0.00064 = \SI{1.1}{cm/s^2} = \SI{0.011}{m/s^2}
\end{align*}
This situation is \emph{mass dominated}, also known as \emph{vibration isolation}.

When the period of the ground acceleration is $T_g=\SI{0.9}{s}$, the maximum values are
\begin{align*}
&\gamma = \frac{0.9}{0.9} = 1 \quad ; \quad
H = \frac{1}{\sqrt{(1-1^2)^2 + 4\cdot 0.05^2\cdot 1^2}} = \frac{1}{\sqrt{0 + 4\cdot 0.05^2}} = 10 \\
&u_0 = 4.1\cdot 10 = \SI{41}{cm} \\
&\ddot{u}_0 = \left(\frac{2\pi}{0.9}\right)^2 \cdot 41 = \SI{2000}{cm/s^2} = \SI{20}{m/s^2}
\end{align*}
The \emph{resonance} situation is \emph{damping dominated}.

And for a period $T_g=\SI{5}{s}$, the maximum values are
\begin{align*}
&\gamma = \frac{0.9}{5} = 0.18 \quad ; \quad
H = \frac{1}{\sqrt{(1-0.18^2)^2 + 4\cdot 0.05^2\cdot 0.18^2}} = 1.033 \\
&u_0 = 4.1\cdot 1.033 = \SI{4.2}{cm} \\
&\ddot{u}_0 = \left(\frac{2\pi}{5}\right)^2 \cdot 4.2 = \SI{6.6}{cm/s^2} = \SI{0.066}{m/s^2}
\end{align*}
Which is a practically \emph{static} situation or \emph{stiffness dominated}.

\end{Answer}
