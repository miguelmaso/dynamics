
\begin{Exercise}[label={two_storey}]
The two storey building shown is supported by four square concrete columns of dimensions $0.35 \times \SI{0.35}{m^2}$. The total masses of the bottom and top floors are $150$ and \qty{100}{Tn} respectively.
\begin{enumerate}[(a)]
    \item Determine the natural modes and frequencies of vibration in the horizontal direction shown.
    \item Determine the frequency of vibration that would be obtained making the assumption that the fundamental mode of vibration increases linearly with height.
\end{enumerate}

\begin{center}
\pictures{twoStorey}
\pictures{storeyPlanView}
\end{center}

\shortAnswer (a) $\omega_1=\SI{10}{rad/s}$, $\omega_2=\SI{24.5}{rad/s}$, $\mathbf{v}_1 = [2\ 3]^T$, $\mathbf{v}_2 = [1\ {-}1]^T$
(b) $\omega_1=\SI{10.4}{rad/s}$, $\mathbf{v}_1 = [1\ 2]^T$
\end{Exercise}



\begin{Answer}[ref={two_storey}]
The inter-storey stiffness is obtained from the concrete columns properties,
\begin{align*}
&E = \SI{14}{GPa} = \SI{14e6}{KPa} \\
&I = \frac{1}{12}bc^3 = \frac{1}{12}0.35^4 = \frac{15}{12}\times10^{-3}\si{m^4} \\
&k_{column} = \frac{12EI}{h^3} = \frac{12\times14\times15\times10^3}{12\times27} = \SI{7.77e3}{KN/m} \\
&k = 4k_{column} = \SI{31.1}{KN/m}
\end{align*}

Using those values, the mass and stiffness matrices are
$$
\mathbf{M} =
\begin{bmatrix}
    m_1 & 0 \\
    0 & m_2
\end{bmatrix} =
\begin{bmatrix}
    150 & 0 \\
    0 & 100
\end{bmatrix} \si{Tn}
$$

$$
\mathbf{K} =
\begin{bmatrix}
    k_1 + k_2 & -k_2\\
    -k_2 & k_2
\end{bmatrix} \approx
\begin{bmatrix}
    60 & -30 \\
    -30 & 30
\end{bmatrix} \times10^3 \si{KN/m}
$$

\paragraph{(a) Natural modes of vibration}
The natural modes of vibration can be obtained from the solution to the generalized eigenvalue problem
\begin{align*}
|\mathbf{K} - \omega^2 \mathbf{M}| &= 0 \\
\left\vert \begin{matrix}
    60000 -\omega^2 150 & -30000 \\
    -30000 & 30000 -\omega^2 100
\end{matrix} \right\vert &= 0 \\
(6000-15\omega^2)(3000-1\omega^2)-3000\times3000 &= 0 \\
15\omega^2 - 10500\omega^2 + 900000 &= 0
\end{align*}
$$
\omega^2 = \begin{cases}
    100 \rightarrow \omega_1 = \SI{10}{rad/s} \\
    600 \rightarrow \omega_2 = \SI{24.5}{rad/s}
\end{cases}
$$

Since the the determinant of the system is zero for the eigenvalues, the eigenvectors are found assigning an arbitrary value to a component,
\begin{align*}
(\mathbf{K} -\omega_1^2\mathbf{M})\mathbf{v}_1 = \mathbf{0} \\
\begin{bmatrix}
    4500 & -3000 \\
    -3000 & 2000
\end{bmatrix} \mathbf{v}_1 = \mathbf{0} \\
\mathbf{v}_1 = \begin{bmatrix}
    2 \\ 3
\end{bmatrix}
\end{align*}

\begin{align*}
(\mathbf{K} -\omega_2^2\mathbf{M})\mathbf{v}_2 = \mathbf{0} \\
\begin{bmatrix}
    -3000 & -3000 \\
    -3000 & -3000
\end{bmatrix} \mathbf{v}_2 = \mathbf{0} \\
\mathbf{v}_2 = \begin{bmatrix}
    1 \\ -1
\end{bmatrix}
\end{align*}


\paragraph{(b) Ritz-Rayleigh method}
In order to show the possibilities of the Ritz-Rayleigh method, a linearly increasing mode of vibration $\mathbf{r}_1$ is choosen. If it were the case of a more complex structure, more trial modes of vibration $\mathbf{r}_i$ could be provided.
\begin{align*}
\mathbf{R} &= [\mathbf{r}_1] = \begin{bmatrix}
    1 \\ 2
\end{bmatrix} \\
\hat{\mathbf{M}} &= \mathbf{R}^T\mathbf{M}\mathbf{R} = \begin{bmatrix}
    1 \\ 2
\end{bmatrix}^T \begin{bmatrix}
    150 & 0 \\
    0 & 100
\end{bmatrix} \begin{bmatrix}
    1 \\
    2
\end{bmatrix} = \SI{550}{Tn} \\
\hat{\mathbf{K}} &= \mathbf{R}^T\mathbf{K}\mathbf{R} = \begin{bmatrix}
    1 \\ 2
\end{bmatrix}^T \begin{bmatrix}
    60000 & -30000 \\
    -30000 & 30000
\end{bmatrix} \begin{bmatrix}
    1 \\
    2
\end{bmatrix} = \SI{6e4}{KN/m}
\end{align*}
And the eigenvalue problem is reduced to a scalar equation,
$$
|\hat{\mathbf{K}} - \omega^2\hat{\mathbf{M}}| = 0
\quad\rightarrow\quad \omega_1^2 = \frac{6\times10^4}{550} \quad\rightarrow\quad \omega_1 = \SI{10.4}{rad/s}
$$

\end{Answer}
