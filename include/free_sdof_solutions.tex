\solution{1}
The natural frequency of a structure is obtained from the solution of the differential equation governing the displacement of a \emph{spring mass} system without excitation.
$$
m\ddot{u}+ku=0
$$
where $m$ is the mass of the idealized system and $k$ is the stiffness. The natural frequency depends on both constants, $\omega^2 = k/m$.

Every single structure can be decomposed in its elements and each element, analyzed by any of the standard methods. Here, to obtain the stiffness of each element, we impose a unit displacement $u_0$ generated by the corresponding force $F_0$. The stiffness of the structure is the sum of the stiffness of its components.

\begin{center}
    \pictures{portico3PinnedSol}
    \pictures{columnPinned}
\end{center}

The displacement of the columns can be analyzed as a cantilever using static analysis concepts:
$$
u_0 = \frac{F_0h^3}{3EI} \quad \rightarrow \quad k_{column} = \frac{3EI}{h^3}
$$
Finally, the stiffness and the frequency of the structure are
$$
k = 3k_{column} = \frac{9EI}{h^3} \quad , \quad \omega = \sqrt{\frac{9EI}{mh^3}}
$$

\begin{center}
    \pictures{porticoSol}
    \pictures{columnFixed}
\end{center}

Analogously, the second structure can be analyzed combining the stiffness of the columns. In that case, rotation $\varphi_0 = u_0/h$ generated by the moment reaction $M_0$ has been imposed to the equivalent beams. The moment reaction must satisfy global equilibrium:
$$
\sum M = 2M_0 -F_0h = 0
$$
And from static analysis, the rotation generated by the moment is
$$
\varphi_0 = \frac{M_0h}{6EI}
$$
Substituting the moment and the rotation into the above expression gives
$$
\frac{u_0}{h} = \frac{F_0h}{12EI} \quad \rightarrow \quad k_{column} = \frac{12EI}{h^3}
$$
The lateral stiffness and frequency of the structure are
$$
k = 2k_{column} = \frac{24EI}{h^3} \quad , \quad \omega = \sqrt{\frac{24EI}{mh^3}}
$$

\begin{center}
    \pictures{porticoBracingSol}
    \pictures{bracing}
\end{center}

The last structure adds two braces and its stiffness shall be added, but only one of them is contributing, since the bracing under compression buckles. The stiffness of a brace is
$$
\delta = \frac{Pl}{EA} \quad \rightarrow \quad k_{brace} = \frac{EA}{l}\cos^2\theta
$$
and the lateral stiffness and frequency of the structure are
$$
k = 2k_{column} + k_{brace} = \frac{24EI}{h^3}+\frac{EA}{l}\cos^2\theta \quad , \quad 
\omega = \sqrt{\frac{24EI}{mh^3}+\frac{EA}{ml}\cos^2\theta}
$$



\solution{2}
The Rayleigh's method is based on assuming the vibration to be given in terms of a pre-determined shape function $\psi$ as $u(x,t) = u_0(t)\psi(x)$. This assumption allows to find an equivalent mass and stiffness associated to that mode of vibration and thus, the frequency.

\parbox{.7\textwidth}{For the simple beam, a possible shape function would be a parabola satisfying the kynematic boundary conditions, $\psi(0) = \psi(l) = 0$.} \hspace{1em}
\parbox{.25\textwidth}{\pictures{simpleBeam}}
\begin{align*}
&\psi(x) = \frac{4}{l^2} x (l-x) \\
&\psi''(x) = \frac{8}{l^2}
\end{align*}
where the term $4/l^2$ is a scaling parameter and does not modify the solution. The equivalent mass and stiffness and frequency are
\begin{align*}
&m = \int_0^l \rho A\psi^2dx = \int_0^l \rho A \left(\frac{4}{l^2} x (l-x)\right)^2dx = \frac{8\rho Al}{15} \\
&k = \int_0^l EI \psi''^2 dx = \int_0^l EI \left(\frac{8}{l^2}\right)^2dx = \frac{64EI}{l^3} \\
&\omega = \sqrt{\frac{k}{m}} = \sqrt{\frac{64EI}{l^3}\frac{15}{8\rho Al}} = \frac{1}{l^2}\sqrt{120\frac{EI}{\rho A}} \approx 10.95\frac{1}{l^2}\sqrt{\frac{EI}{\rho A}} rad/s
\end{align*}
The obtained result can be compared against the exact frequency, obtained by the same procedure applied to a fourth order polynomial satisfying both kinematic and dynamic boundary conditions. The analytical result is $\omega = \frac{1}{l^2}\sqrt{\frac{3024EI}{31\rho A}} \approx 9.86\frac{1}{l^2}\sqrt{\frac{EI}{\rho A}} rad/s$.

\parbox{.7\textwidth}{The cantilever can be analyzed using a second order polynomial. In that case, the kinematic boundary conditions are $\psi(0) = \psi'(0) = 0$.} \hspace{1em}
\parbox{.25\textwidth}{\pictures{cantilever}}
\begin{align*}
&\psi(x) = \frac{1}{l^2} x^2 \\
&\psi''(x) = \frac{2}{l^2}
\end{align*}
Again, the term $1/l^2$ is a scaling term providing consistency and does not modify the result of the calculations.
\begin{align*}
&m = \int_0^l \rho A\psi^2dx = \int_0^l \rho A \left(\frac{1}{l^2} x^2\right)^2dx = \frac{\rho Al}{5} \\
&k = \int_0^l EI \psi''^2 dx = \int_0^l EI \left(\frac{2}{l^2}\right)^2dx = \frac{4EI}{l^3} \\
&\omega = \sqrt{\frac{k}{m}} = \sqrt{\frac{4EI}{l^3}\frac{5}{\rho Al}} = \frac{1}{l^2}\sqrt{20\frac{EI}{\rho A}} \approx 4.47\frac{1}{l^2}\sqrt{\frac{EI}{\rho A}} rad/s
\end{align*}
On the other hand, the exact frequency of the first mode of vibration of a cantilever is $\omega\approx3.52\frac{1}{l^2}\sqrt{\frac{EI}{\rho A}}rad/s$.

The last two structures can be analyzed using a cubic polynomial, since the shape function must fulfill three kinematic boundary conditions.

For the application of Rayleigh's method with periodic shape functions, see the solution to example 9.



\solution{3}
Given that the natural period of vibration of the frame is $T=0.9$ seconds, the frequency is computed as
$$
\omega = \frac{2\pi}{T} = \frac{2\pi}{0.9} = 6.98rad/s
$$
Then, the lateral stiffness is computed from the frequency and the mass of the structure,
$$
\omega^2 = \frac{k}{m} \quad \rightarrow \quad
k = \omega^2m = \omega^2\frac{P}{g} = 6.98^2\frac{100}{10} = 487KN/m
$$

The goal of the second step is to determine the diameter of the steel cross-braces required to strengthen the structure by reducing the period to 0.3 seconds. First of all, the new stiffness is obtained following the same procedure,
\begin{align*}
\omega = \frac{2\pi}{T} = \frac{2\pi}{0.3} = 20.94rad/s \\
k = \omega^2\frac{P}{g} = 20.94^2\frac{100}{10} = 4384KN/m
\end{align*}
The bracing system should provide the additional stiffness,
$$
k = k_{frame} + k_{bracing} \quad \rightarrow \quad k_{bracing} = 4384 - 487 = 3897KN/m
$$

We will consider the stiffness of one brace because the one under compression can buckle. 
Following the solution of example 1 (c), the area can be obtained from
\begin{align*}
k_{brace} &= \frac{EA}{l}\cos^2\theta \quad \rightarrow \quad A = \frac{k_{brace}l}{E\cos^2\theta} \\
l &= \sqrt{4^2 + 5^2} = 6.4m \\
\cos\theta &= \frac{5}{6.4} = 0.78 \\
A &= \frac{3897\cdot 6.4}{2\cdot 10^8\cdot 0.78} = 1.6\cdot 10^{-4} m^2 = 1.6cm^2
\end{align*}
and the diameter is obtained from the area,
$$
D = 2\sqrt{\frac{A}{\pi}} = 2\sqrt{\frac{1.6}{\pi}} = 1.4cm
$$

The last part of the example consists on computing the new period of vibration if a further load of $50KN$ is added to the strengthened structure. Using the new lateral stiffness, the frequency is
$$
\omega = \sqrt{\frac{k}{m}} = \sqrt{\frac{4384}{15}} = 17.1 rad/s
$$
and the new period is
$$
T = \frac{2\pi}{\omega} = \frac{2\pi}{17.1} = 0.37s
$$



\solution{4}
The solution to the differential equation governing free vibration of a \emph{mass spring damper} system is governed by the natural and damped frequencies and the damping ratio:
$$
u(t) = e^{-\xi\omega t}\sin(\omega_D t) \quad ; \quad \omega_D = \omega\sqrt{1-\xi^2}
$$
For simplicity, we will assume a small damping and $\sqrt{1-\xi^2}\rightarrow1$, thus, the natural period can be obtained directly from the measured values
$$
T \approx T_D = \frac{3}{20} = 0.15s
$$
And the damping ratio $\xi$ is directly related to the decay coefficient determined by the relation among two consecutive oscillations from a damped period
$$
\frac{u(t)}{u(t+T_D)} = e^{\xi\omega T_D} = e^{\frac{2\pi\xi}{\sqrt{1-\xi^2}}} \quad \rightarrow \quad
\frac{u_0}{u_1} \frac{u_1}{u_2} \dots \frac{u_{n-1}}{u_n} = \frac{u_0}{u_n} = e^{\frac{2n\pi\xi}{\sqrt{1-\xi^2}}}
$$
Since $\xi$ is small, the damping ratio can be found as
\begin{align*}
2n\pi\xi = \log\left(\frac{u_0}{u_n}\right) \quad \rightarrow \quad
\xi = \frac{1}{2n\pi} \log\left(\frac{u_0}{u_n}\right) = \frac{1}{2\cdot20\pi} \log\left(\frac{5}{1}\right) = 0.0128 \\
\xi = 1.28\%
\end{align*}
