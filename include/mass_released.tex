
\begin{Exercise}[label={mass_released}]
A mass $m$ is released from a given height attached to a massless cable of length $l$, area $A$ and Young's modulus $E$. If the cable is fixed at the point from which the mass is released, describe the motion/vibration of the mass. Determine the maximum stress in the cable and the lowest point reached by the mass.

\begin{center}
\pictures{masslessCable}
\end{center}

\shortAnswer $u_{max} = 2\frac{mg}{EA}$, $\sigma_{max} = 2\frac{mg}{Al}$  
\end{Exercise}



\begin{Answer}[ref={mass_released}]
The motion of the mass released from a given height and attached to a cable has two parts. Firstly, a free fall. Secondly, a vibration. The velocity at the end of the free fall is
$$
\frac{1}{2}m\dot{u}_0^2 = mgl \quad \rightarrow \quad \dot{u}_0 = \sqrt{gl}
$$

And the vibration is described by a sudden force suddenly applied; the external force $F_0$ is generated by the gravity weight $mg$ and the stresses in the cable are obtained from elastic analysis. The same equations of example 7 can be applied here. Neglecting damping,
$$
u(t) = \frac{F_0}{k}(1 -\cos(\omega t))
$$

And the maximum displacement $u$ and stress $\sigma$ are
\begin{align*}
&u_{max} = \frac{F_0}{k}2 = 2\frac{mg}{EA} \\
&\sigma_{max} = E\varepsilon_{max} = E\frac{u_{max}}{l} = 2\frac{mg}{Al}
\end{align*}
\end{Answer}
