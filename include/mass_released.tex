
\begin{Exercise}[label={mass_released}]
A mass $m$ is released from a given height attached to a massless cable of length $l$, area $A$ and Young's modulus $E$. If the cable is fixed at the point from which the mass is released, describe the motion/vibration of the mass. Determine the maximum stress in the cable and the lowest point reached by the mass.

\begin{center}
\pictures{masslessCable}
\end{center}

\shortAnswer $u_{max} = \left(\frac{mg}{EA} + \sqrt{\frac{2mg}{EA}}\right)l$, $\sigma_{max} = \frac{mg}{A} + \sqrt{\frac{2mgE}{A}}$.
\end{Exercise}



\begin{Answer}[ref={mass_released}]
The motion of the mass released from a given height and attached to a cable has two parts: first, a free fall; second, a vibration. The velocity at the end of the free fall is
$$
\frac{1}{2}m\dot{u}_0^2 = mgl \quad \rightarrow \quad \dot{u}_0 = \sqrt{gl}
$$
and may be taken as the initial condition of the vibration. Lastly, the elastic response of the cable is the superposition of the static elongation and the vibration. From the cable properties,
\begin{align*}
k &= \frac{EA}{l} \\
m &= \sqrt{\frac{k}{m}} = \sqrt{\frac{EA}{ml}} \\
u_{stat} &= \frac{F}{M} = \frac{mgl}{EA}\ .
\end{align*}

Then, neglecting damping, we can consider the motion as the transient solution to the differential equation satisfying the appropriate initial conditions,
\begin{align*}
u(t) &= C\sin(\omega t + \phi) + u_{stat} \\
u(0) &= u_{stat} \rightarrow C\sin(\phi) = 0 \rightarrow \phi = 0 \\
\dot{u}(0) &= \sqrt{2gl} \rightarrow C\omega\cos(\phi) =  \sqrt{2gl} \rightarrow C = \sqrt{\frac{2mg}{EA}}l
\end{align*}

and the maximum displacement $u_{max}$ and stress $\sigma$ are
\begin{align*}
u_{max} &= \left(\frac{mg}{EA} + \sqrt{\frac{2mg}{EA}}\right)l \\
\sigma &= E\epsilon = E\frac{u_{max}}{l} = \frac{mg}{A} + \sqrt{\frac{2mgE}{A}}\ .
\end{align*}

N.B. the maximum displacement can also be obtained by applying energy conservation.
\end{Answer}
