
\begin{Exercise}[label={helicopter_impulse}]
A building has a height of $100m$, a square base measuring $20\times20m^2$, an average specific weight of $1\,500N/m^3$ and a natural period of vibration of 5 seconds. The top floor is hit by an helicopter with a mass of $10\,000Kg$ and traveling at $30m/s$. Determine the maximum deflection at the top assuming conservation of linear momentum and a vibration shape function that increases linearly with the height.

\begin{center}
\pictures{buildingHelicopter}
\end{center}

\shortAnswer $u_0 = 12cm$
\end{Exercise}



\begin{Answer}[ref={helicopter_impulse}]
When the building is hit by the helicopter, the impulse or linear momentum is preserved. To compute the linear momentum of the building, we need to know the equivalent mass and we can use the Rayleigh's method for that purpose. As it is suggested, we will use a linear shape function
\begin{align*}
u& = u_0\psi \quad ; \quad \psi = \frac{z}{h} \\
m& \begin{multlined}[t]= \int_0^h \rho A\psi^2dz = \rho A\int_0^h\frac{z^2}{h^2}dz = \frac{1}{3}\rho Ah = \frac{1}{3} 1500 \cdot 400 \cdot 100 = \\
    = 2\cdot 10^7N = 2\cdot 10^6Kg \end{multlined}
\end{align*}

The impulse of the helicopter is
$$
I = (mv)_{helicopter} = 10^4 \cdot 30 = 3\cdot 10^5 Kg\,m\,s^-1
$$
which is transferred to the building. Then, the maximum deflection at the top of the building is inferred
\begin{align*}
&\dot{u}_0 = \frac{I}{m} \quad \rightarrow \quad u_0 = \frac{I}{m\omega} \\
&\omega = \frac{2\pi}{T} = \frac{2\pi}{5} = 1.25rad\,s^{-1} \\
&u_0 = \frac{3\cdot 10^5}{2\cdot 10^6\cdot 1.25} = 0.12m = 12cm
\end{align*}
\end{Answer}
