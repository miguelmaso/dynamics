
\begin{Exercise}[label={moving_load}]
A point load $P=\SI{1}{kN}$ moves along with constant speed $v=\SI{10}{m/s}$ on a simply supported beam of length $l=10\pi\,\si{m}$, as shown in the figure. The beam is made of concrete, has a rectangular section of \SI{1}{m} width and \SI{0.5}{m} height and an average density of \SI{2800}{kg/m^3}. Determine the deflection of the beam as a function of time, the dynamic magnification factor and the maximum bending moment at the centre section.

\begin{center}
\pictures{beamPointLoad}
\end{center}

\shortAnswer $H=1.11$, $u=0.5\sin(t)\,\si{cm}$, $M_{max} = 8.7\,\si{kN.m}$. \\
References: \cite[p. 305]{chopra}
\end{Exercise}



\begin{Answer}[ref={moving_load}]
In this case, the magnitude of the external force $P=\SI{1}{kN}$ does not vary with time but moves with constant speed $v=\SI{10}{m/s}$. The Rayleigh method allows us to compute the generalized force applied as a function of time. For this purpose, we choose a periodic shape function $\psi$ satisfying the kinematic boundary conditions
\begin{align*}
&\psi = \sin\left(\frac{\pi x}{l}\right) \\
&F = P\psi(x) = P\psi(vt) = P\sin\left(\frac{\pi vt}{l}\right)\ .
\end{align*}
Thus, the period of the external excitation is $\Omega = \pi v/l = \SI{1}{rad/s}$.

From the structure dimensions, we know that the mechanical properties of the section are:
\begin{align*}
&A = \SI{0.5}{m^2} \\
&I = \frac{1}{12}0.5^3 = \frac{0.125}{12} \approx \SI{0.01}{m^4}\\
&E = \SI{14}{GPa}
\end{align*}
and the generalized mass and stiffness can be computed with the Rayleigh method,
\begin{align*}
m& \begin{multlined}[t]= \int_0^l \rho A\psi^2dx = \int_0^l \rho A\sin^2\left(\frac{\pi x}{l}\right)dx = \frac{\rho Al}{2} = \\
\phantom{\hspace{10em}}= \frac{2800 \cdot 0.5 \cdot 10\pi}{2} = 7000\pi Kg = 7\pi \,\si{t}\end{multlined} \\
k& \begin{multlined}[t]= \int_0^l EI\psi''^2dx = \int_0^l EI\left(-\left(\frac{\pi}{l}\right)^2\sin\left(\frac{\pi x}{l}\right)\right)^2dx = \frac{EI\pi^4}{2l^3} = \qquad \\
= \frac{14\cdot 10^6 \cdot 0.01 \cdot \pi^4}{2 \cdot (10\pi)^3} =
70\pi \,\si{kN/m} \end{multlined}
\end{align*}
with frequency equal to
$$
\omega = \sqrt{\frac{k}{m}} = \sqrt{\frac{70\pi}{7\pi}} = \sqrt{10} = \SI{3.16}{rad/s}
$$

Finally, the structural response is obtained following the same steps as in Exercise \ref{frame_ground_acceleration}, a structure under periodic loading. The deflection depends on the dynamic magnification factor $H$ and assuming a small damping ratio:
\begin{align*}
\gamma& = \frac{\Omega}{\omega} = \frac{\pi v}{l\omega} \\
H& \begin{multlined}[t]= \frac{1}{\sqrt{\left(1-\frac{\pi^2v^2}{l^2\omega^2}\right)^2 + 4\xi^2\frac{\pi^2v^2}{l^2\omega^2}}} \approx \frac{1}{1-\frac{\pi^2v^2}{l^2\omega^2}} = \frac{l^2\omega^2}{l^2\omega^2-\pi^2v^2} = \qquad \\
= \frac{10^2\cdot\pi^2\cdot 3.16^2}{10^2\cdot\pi^2\cdot 3.16^2 - 10^2\cdot\pi^2}
= \frac{10}{10-1} = 1.11 \end{multlined}\ .
\end{align*}

Thus, the deflection of the beam as a function of time is
$$
u = \frac{P}{k}H\sin(\Omega t) = \frac{1}{70\pi}1.11\sin(t) = 0.005\sin(t)\,f\si{m} = 0.5\sin(t)\,\si{cm}
$$

And the maximum bending moment at the centre section is
$$
M_{max} = \frac{Pl}{4}H = \frac{1 \cdot 10\pi}{4}1.11 = \SI{8.7}{kN.m}
$$

\paragraph{Solution to the differential equation}
Given that the load is a piecewise function in time, the dynamic magnification factor $H$ may not be accurate. The full solution to the differential equation is also a piecewise function of the form
$$
u =
\begin{cases}
C_1\sin(\omega t + \varphi_1) + \frac{P}{k}H\sin(\Omega t + \phi - \Delta\phi) \quad &\text{if}\ t < \frac{l}{v} \\
C_2\sin(\omega t + \varphi_2) \quad &\text{if}\ t \geq \frac{l}{v}
\end{cases}
$$

The first part of the solution must satisfy homogeneous boundary conditions
$$
\begin{array}{l}
u_1(0) = 0 \\
\dot{u}_1(0) = 0
\end{array} \rightarrow
\begin{array}{l}
\varphi_1 = 0 \\
C_1 = -\frac{P}{k}H\gamma
\end{array}
$$
with $\gamma = \frac{\Omega}{\omega} = \frac{\pi v}{\omega l}$. For the second part of the motion, the expression must satisfy continuity at $t = l/v$,
\begin{align*}
u_2(l/v) = C_2\sin\left(\frac{\omega l}{v}+\varphi_2\right) &= u_1(l/v) = -\frac{P}{k}H\gamma\sin\frac{\omega l}{v} \\
\dot{u}_2(l/v) = C_2\omega\cos\left(\frac{\omega l}{v}+\varphi_2\right) &= \dot{u}_1(l/v) = -\frac{P}{k}H\gamma\omega\cos\frac{\omega l}{v} - \frac{P}{k}H\Omega
\end{align*}
Note that $\Omega l/v = \pi$.

The values of the integration constants are obtained after applying some trigonometric identities:
\begin{gather*}
\omega\cot\left(\frac{\omega l}{v}+\varphi_2\right) = \omega\cot\frac{\omega l}{v} + \omega\csc(\omega l/v) \\
\cot\left(\frac{\omega l}{v}+\varphi_2\right) = \cot\frac{\omega l}{2v} \\
\varphi_2 = -\frac{\omega l}{2v}
\end{gather*}
and
\begin{gather*}
C_2 = -\frac{P}{k}H\gamma\frac{\sin(\omega l/v)}{\sin(\omega l/v + \varphi_2)} =
    -\frac{P}{k}H\gamma\frac{\sin(\omega l/v)}{\sin(\omega l/v - \omega l/(2v))} \\[4pt]
C_2 = -\frac{P}{k}H\gamma\frac{2\sin(\omega l/(2v))\cos(\omega l/(2v))}{\sin(\omega l/(2v))} \\[2pt]
C_2 = -2\frac{P}{k}H\gamma\cos\frac{\omega l}{2v}
\end{gather*}

The result is
$$
u = \frac{P}{k}\frac{\omega^2l^2}{\omega^2l^2-\pi^2v^2}
\begin{cases}
\sin\frac{\pi vt}{l} - \frac{\pi v}{\omega l}\sin(\omega t) \quad &\text{if}\ t < \frac{l}{v} \\
-\frac{2\pi v}{\omega l}\cos\frac{\omega l}{2v}\sin(\omega t - \frac{\omega l}{2v}) \quad &\text{if}\ t \geq \frac{l}{v}
\end{cases}
$$

\end{Answer}
