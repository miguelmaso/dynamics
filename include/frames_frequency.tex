
\begin{Exercise}[label={frames_frequency}]
For the structures shown, determine the natural frequency of vibration using simple structural concepts.
Consider the frames below. Compute the stiffness and natural frequencies.

\begin{center}
    \pictureslabel{portico3Pinned}{a}
    \hspace{2em}
    \pictureslabel{portico}{b}
    \hspace{2em}
    \pictureslabel{porticoBracing}{c}
\end{center}

\shortAnswer(a) $\omega = \sqrt{\frac{9EI}{mh^3}}$, (b) $\omega = \sqrt{\frac{24EI}{mh^3}}$ and (c) $\omega = \sqrt{\frac{24EI}{mh^3}+\frac{EA}{ml}\cos^2\theta}$.
\end{Exercise}



\begin{Answer}[ref={frames_frequency}]
The natural frequency of a structure is obtained from the solution of the differential equation governing the displacement of a \emph{spring mass} system without excitation.
$$
m\ddot{u}+ku=0
$$
where $m$ is the mass of the idealized system and $k$ is the stiffness. The natural frequency depends on both constants, $\omega^2 = k/m$.

Every single structure can be decomposed in its elements and each element, analyzed by any of the standard methods. Here, to obtain the stiffness of each element, we impose a unit displacement $u_0$ generated by the corresponding force $F_0$. The stiffness of the structure is the sum of the stiffness of its components.

\begin{center}
    \pictures{portico3PinnedSol}
    \pictures{columnPinned}
\end{center}

The displacement of the columns can be analyzed as a cantilever using static analysis concepts:
$$
u_0 = \frac{F_0h^3}{3EI} \quad \rightarrow \quad k_{column} = \frac{3EI}{h^3}
$$
Finally, the stiffness and the frequency of the structure are
$$
k = 3k_{column} = \frac{9EI}{h^3} \quad , \quad \omega = \sqrt{\frac{9EI}{mh^3}}
$$

\begin{center}
    \pictures{porticoSol}
    \pictures{columnFixed}
\end{center}

Analogously, the second structure can be analyzed combining the stiffness of the columns. In that case, rotation $\varphi_0 = u_0/h$ generated by the moment reaction $M_0$ has been imposed to the equivalent beams. The moment reaction must satisfy global equilibrium:
$$
\sum M = 2M_0 -F_0h = 0
$$
And from static analysis, the rotation generated by the moment is
$$
\varphi_0 = \frac{M_0h}{6EI}
$$
Substituting the moment and the rotation into the above expression gives
$$
\frac{u_0}{h} = \frac{F_0h}{12EI} \quad \rightarrow \quad k_{column} = \frac{12EI}{h^3}
$$
The lateral stiffness and frequency of the structure are
$$
k = 2k_{column} = \frac{24EI}{h^3} \quad , \quad \omega = \sqrt{\frac{24EI}{mh^3}}
$$

\begin{center}
    \pictures{porticoBracingSol}
    \pictures{bracing}
\end{center}

The last structure adds two braces and its stiffness shall be added, but only one of them is contributing, since the bracing under compression buckles. The stiffness of a brace is
$$
\delta = \frac{Pl}{EA} \quad \rightarrow \quad k_{brace} = \frac{EA}{l}\cos^2\theta
$$
and the lateral stiffness and frequency of the structure are
$$
k = 2k_{column} + k_{brace} = \frac{24EI}{h^3}+\frac{EA}{l}\cos^2\theta \quad , \quad 
\omega = \sqrt{\frac{24EI}{mh^3}+\frac{EA}{ml}\cos^2\theta}
$$
\end{Answer}
