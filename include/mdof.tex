\section{Vibration of MDOF structures}


\example{15}
The two storey building shown is supported by four square concrete columns of dimensions $0.35 \times 0.35\ m^2$. The total masses of the bottom and top floors are $150$ and $100\ Tn$ respectively.
\begin{enumerate}[(a)]
    \item Determine the natural modes and frequencies of vibration in the horizontal direction shown.
    \item Determine the frequency of vibration that would be obtained making the assumption that the fundamental mode of vibration increases linearly with height.
\end{enumerate}

\begin{center}
\pictures{twoStorey}
\pictures{storeyPlanView}
\end{center}



\example{16}
The two storey building from the previous exercise is hit by a helicopter with a mass of $10\ Tn$ traveling at $20\ m/s$.
\begin{enumerate}[(a)]
    \item Determine the resulting vibration and the maximum displacement at the top of the building using both modes of vibration.
    \item Determine the resulting vibration and the maximum displacement at the top on the assumption that the linearly increasing mode absorbs the total momentum.
\end{enumerate}



\example{17}
A three storey building has the mass and stiffness distribution shown.

\begin{enumerate}[(a)]
    \item Approximate the first period of vibration using a linearly increasing mode.
    \item Using a linearly increasing mode together with a second Ritz vector increasing quadratically with height, approximate the first two modes and frequencies of vibration.
\end{enumerate}    

\begin{center}
\pictures{threeStoreyKM}
\end{center}
