\section{Forced vibration of SDOF structures}


\example{5} The portal frame of example 3 (a) is subject to a sinusoidal ground vibration with horizontal acceleration amplitude of $2m/s^2$. Assuming a damping ratio of $5\%$, determine the maximum displacement and maximum total acceleration of the frame when the period of floor vibration is: (a) 0.1 seconds; (b) 0.9 seconds and (c) 5 seconds.

\begin{center}
\pictures{porticoForced}
\end{center}

Solutions: (a) $u_0 = 0.00064cm$, $\ddot{u}_0 = 0.011m\,s^{-1}$; (b) $u_0 = 41cm$, $\ddot{u}_0 = 20m\,s^{-1}$ and (c) $u_0 = 4.2cm$, $\ddot{u}_0 = 0.066m\,s^{-1}$



\example{6} A building has a height of $100m$, a square base measuring $20\times20m^2$, an average specific weight of $1\,500N/m^3$ and a natural period of vibration of 5 seconds. The top floor is hit by an helicopter with a mass of $10\,000Kg$ and traveling at $30m/s$. Determine the maximum deflection at the top assuming conservation of linear momentum and a vibration shape function that increases linearly with the height.

\begin{center}
\pictures{buildingHelicopter}
\end{center}

Solution: $u_0 = 12cm$



\example{7} The building of example 6 is hit by a sudden wind gust which results in the sudden application of horizontal forces distributed along the height of the building as shown in the picture. Assuming a vibration shape function that increases linearly with the height and neglecting damping, determine the maximum displacement at the top of the building.

\begin{center}
\pictures{buildingWind}
\end{center}

Solution: $u_0 = 2.5cm$



\example{8} A mass $m$ is released from a given height attached to a massless cable of length $l$, area $A$ and Young's modulus $E$. If the cable is fixed at the point from which the mass is released, describe the motion/vibration of the mass. Determine the maximum stress in the cable and the lowest point reached by the mass.

\begin{center}
\pictures{masslessCable}
\end{center}

Solution: $u_{max} = 2\frac{mg}{EA}$, $\sigma_{max} = 2\frac{mg}{Al}$



\example{9} A point load $F=1KN$ moves along with constant speed $v=10m/s$ on a simply supported beam of length $l=1p\pi m$ as shown in the figure. The beam is made of concrete, has a rectangular section of $1m$ width and $0.5m$ height and an average density of $2\,800Kg/m^3$. Determine the deflection of the beam as a function of time, the dynamic magnification factor and the maximum bending moment at the centre section.

\begin{center}
\pictures{beamPointLoad}
\end{center}

Solution: $H=1.11$, $u=0.5\sin(t)\,cm$, $M_{max} = 8.7\,KNm$


\example{10} A concrete ribbed slab floor spans 9 m and has an average mass of $500 kg/m^2$. The floor is simply supported on either side and has a natural frequency of vibration of $6.3 Hz$. The floor is to be used for aerobics and other similar rhythmic activities at frequencies ranging from $1.5 Hz$ to $2.5 Hz$ and with contact ratios $\alpha$ between 0.5 and 1. During these activities the average imposed load will remain below 0.75 $kN/m^2$ (before dynamic magnification) and the damping ratio can be taken to be $3\%$.

\begin{enumerate}[(a)]
    \item Determine the maximum possible resonant displacement and the resulting peak acceleration and bending moment per unit width.
    \item If the floor has been designed for a service load of $5kN/m^2$, determine its suitability for the proposed use.
\end{enumerate}

\begin{center}
\pictures{beamPeople}
\hspace{1em}
\pictures{contactRatio}
\end{center}
