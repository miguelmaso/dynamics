
\begin{Exercise}[label={free_vibration}]
In order to determine the dynamic properties of a simply supported bridge with a mass of \SI{e6}{Kg}, the midpoint is displaced \SI{5}{mm} by a jack and then suddenly released. At the end of 20 complete cycles, the time is 3 seconds and the peak displacement measured is \SI{1}{mm}. Determine the natural period and damping ratio of the bridge.

\begin{center}
\pictures{bridge}
\end{center}

\shortAnswer $T=\SI{0.15}{s}$, $\xi=\SI{1.28}{\%}$.v \\
References: \cite[page 49]{chopra}; \cite[page 287]{blanco}
\end{Exercise}



\begin{Answer}[ref={free_vibration}]
The solution to the differential equation governing free vibration of a \emph{mass spring damper} system is governed by the natural and damped frequencies and the damping ratio:
$$
u(t) = e^{-\xi\omega t}\sin(\omega_D t) \quad ; \quad \omega_D = \omega\sqrt{1-\xi^2}
$$
For simplicity, we will assume a small damping and $\sqrt{1-\xi^2}\rightarrow1$, thus, the natural period can be obtained directly from the measured values
$$
T \approx T_D = \frac{3}{20} = \SI{0.15}{s}
$$
And the damping ratio $\xi$ is directly related to the decay coefficient determined by the relation among two consecutive oscillations from a damped period
$$
\frac{u(t)}{u(t+T_D)} = e^{\xi\omega T_D} = e^{\frac{2\pi\xi}{\sqrt{1-\xi^2}}} \quad \rightarrow \quad
\frac{u_0}{u_1} \frac{u_1}{u_2} \dots \frac{u_{n-1}}{u_n} = \frac{u_0}{u_n} = e^{\frac{2n\pi\xi}{\sqrt{1-\xi^2}}}
$$
Since $\xi$ is small, the damping ratio can be found as
\begin{align*}
2n\pi\xi = \log\left(\frac{u_0}{u_n}\right) \quad \rightarrow \quad
\xi = \frac{1}{2n\pi} \log\left(\frac{u_0}{u_n}\right) = \frac{1}{2\cdot20\pi} \log\left(\frac{5}{1}\right) = 0.0128 \\
\xi = \SI{1.28}{\%}
\end{align*}    

\end{Answer}