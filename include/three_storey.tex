
\begin{Exercise}[label={three_storey_helicopter}]
A three storey building has the mass and stiffness distribution shown.

\begin{enumerate}[(a)]
    \item Approximate the first period of vibration using a linearly increasing mode.
    \item Using a linearly increasing mode together with a second Ritz vector increasing quadratically with height, approximate the first two modes and frequencies of vibration.
\end{enumerate}    

\begin{center}
\pictures{threeStoreyKM}
\end{center}
\shortAnswer
\end{Exercise}



\begin{Answer}[ref={three_storey_helicopter}]
The first step before using the Ritz-Rayleigh method to approximate the periods of vibration is to construct the mass and stiffness matrices:
$$
\mathbf{M} = \begin{bmatrix}
    m_1 & 0 & 0 \\
    0 & m_2 & 0 \\
    0 & 0 & m_3
\end{bmatrix}
= \begin{bmatrix}
    2 & 0 & 0 \\
    0 & 1.5 & 0 \\
    0 & 0 & 1
\end{bmatrix} 10^5 kg
$$
$$
\mathbf{K} = \begin{bmatrix}
    k_1+k_2 & -k_2 & 0 \\
    -k_2 & k_2+k_3 & -k_3 \\
    0 & -k_3 & k_3
\end{bmatrix}
= \begin{bmatrix*}[r]
    5 & -2 & 0 \\
    -2 & 3 & -1 \\
    0 & -1 & 1
\end{bmatrix*} 10^7 N/m
$$

\paragraph{1 mode} We start applying the Ritz-Rayleigh method with a mode linearly increasing with height. The matrix of deformation modes will contain only one vector:
$$
\mathbf{R} = [\mathbf{r}_1] =
\begin{bmatrix}
    1 & 2 & 3
\end{bmatrix}^T
$$

And the reduced system is obtained by the following arithmetics:
$$
\hat{\mathbf{M}} = \mathbf{R}^T \mathbf{M} \mathbf{R} =
\begin{bmatrix}
    1 & 2 & 3
\end{bmatrix}
\begin{bmatrix}
    2 & 0 & 0 \\
    0 & 1.5 & 0 \\
    0 & 0 & 1
\end{bmatrix}
\begin{bmatrix}
    1 \\ 2 \\ 3
\end{bmatrix} =
\begin{bmatrix}
    2 & 3 & 3
\end{bmatrix}
\begin{bmatrix}
    1 \\ 2 \\ 3
\end{bmatrix} = 17 \cdot 10^5 kg
$$
$$
\hat{\mathbf{K}} = \mathbf{R}^T \mathbf{K} \mathbf{R} =
\begin{bmatrix}
    1 & 2 & 3
\end{bmatrix}
\begin{bmatrix}
    5 & -2 & 0 \\
    -2 & 3 & -1 \\
    0 & -1 & 1
\end{bmatrix}
\begin{bmatrix}
    1 \\ 2 \\ 3
\end{bmatrix} =
\begin{bmatrix}
    1 & 1 & 1
\end{bmatrix}
\begin{bmatrix}
    1 \\ 2 \\ 3
\end{bmatrix} = 6 \cdot 10^7 N/m
$$

Thus, the eigenvalue problem is reduced to a scalar equation:
$$
\vert \hat{\mathbf{K}} - \omega^2\hat{\mathbf{M}} \vert = 0 \quad \rightarrow \quad
\omega^2_1 = \frac{6\cdot10^7}{17\cdot10^5} = 35.3 \quad \rightarrow \quad
\omega_1 = 5.9\ rad\ s^{-1}
$$


\paragraph{2 modes} The Ritz-Rayleigh method can be enriched with more modes of vibration, in this case, with a mode varying quadratically with height,
$$
\mathbf{R} =
\begin{bmatrix}
    r_1 & r_2
\end{bmatrix} =
\begin{bmatrix}
    1 & 2 & 3 \\
    1 & 4 & 9
\end{bmatrix}^T
$$

The reduced system is then obtained by the algebraic multiplications
\begin{multline*}
\hat{\mathbf{M}} = \mathbf{R}^T \mathbf{M} \mathbf{R} =
\begin{bmatrix}
    1 & 2 & 3 \\
    1 & 4 & 9
\end{bmatrix}
\begin{bmatrix}
    2 & 0 & 0 \\
    0 & 1.5 & 0 \\
    0 & 0 & 1
\end{bmatrix}
\begin{bmatrix}
    1 & 1 \\
    2 & 4 \\
    3 & 9
\end{bmatrix} =
\begin{bmatrix}
    2 & 3 & 3 \\
    2 & 6 & 9
\end{bmatrix}
\begin{bmatrix}
    1 & 1 \\
    2 & 4 \\
    3 & 9
\end{bmatrix} = \\
= \begin{bmatrix}
    17 & 41 \\
    41 & 107
\end{bmatrix} \cdot 10^5 kg
\end{multline*}
\begin{multline*}
\hat{\mathbf{K}} = \mathbf{R}^T \mathbf{K} \mathbf{R} =
\begin{bmatrix}
    1 & 2 & 3 \\
    1 & 4 & 9
\end{bmatrix}
\begin{bmatrix}
    5 & -2 & 0 \\
    -2 & 3 & -1 \\
    0 & -1 & 1
\end{bmatrix}
\begin{bmatrix}
    1 & 1 \\
    2 & 4 \\
    3 & 9
\end{bmatrix} =
\begin{bmatrix}
    1 & 1 & 1
\end{bmatrix}
\begin{bmatrix}
    1 & 1 \\
    2 & 4 \\
    3 & 9
\end{bmatrix} = \\
= \begin{bmatrix}
    6 & 14 \\
    14 & 46
\end{bmatrix} \cdot 10^7 N/m
\end{multline*}

and the eigenvalue problem involves the calculation of the determinant of a $2\times2$ matrix,
$$
\vert \hat{\mathbf{K}} - \omega^2 \hat{\mathbf{M}} \vert =
\left\vert \begin{matrix}
    600 -\omega^2 17 & 1400 -\omega^2 41 \\
    1400 -\omega^2 41 & 4600 -\omega^2 107
\end{matrix}
\right\vert \cdot 10^5 = 0
$$

The result is the second order polynomial
$
138 \omega^4 - 27600 \omega^2 + 800000 = 0
$ with the following roots:
\begin{align*}
\omega_1^2& = 35.1 \quad \rightarrow \quad \omega_1 = 5.9\ rad\ s^{-1} \\
\omega_2^2& = 164.8 \quad \rightarrow \quad \omega_2 = 12.8\ rad\ s^{-1} \\
\end{align*}


\paragraph{Modes of vibration} The Ritz-Rayleigh method can also be used to find accurate eigenvectors.
    
\end{Answer}
