
\begin{Exercise}[label={rayleigh_sdof}]
For the structures shown, determine the natural frequency of vibration using the Rayleigh method.

\begin{center}
    \pictureslabel{simpleBeam}{a}
    \pictureslabel{nSpanBeam}{b} \\
    \pictureslabel{cantilever}{c}
    \pictureslabel{fixedPinnedBeam}{d}
\end{center}

\shortAnswer (a) $\omega \approx 10.95\frac{1}{l^2}\sqrt{\frac{EI}{\rho A}}\,\si{rad/s}$, (c) $\omega \approx 4.47\frac{1}{l^2}\sqrt{\frac{EI}{\rho A}}\,\si{rad/s}$.
\end{Exercise}


\begin{Answer}[ref={rayleigh_sdof}]
The Rayleigh method is based on assuming the vibration to be given in terms of a pre-determined shape function $\psi$ as $u(x,t) = u_0(t)\psi(x)$. This assumption allows you to find an equivalent mass and stiffness associated with that mode of vibration and thus, the frequency.

\parbox{.7\textwidth}{For the simple beam, a possible shape function would be a parabola satisfying the kinematic boundary conditions, $\psi(0) = \psi(l) = 0$.} \hspace{1em}
\parbox{.25\textwidth}{\pictures{simpleBeam}}
\begin{align*}
&\psi(x) = \frac{4}{l^2} x (l-x) \\
&\psi''(x) = \frac{8}{l^2}
\end{align*}
where the term $4/l^2$ is a scaling parameter and does not modify the solution. The equivalent mass and stiffness and frequency are
\begin{align*}
&m = \int_0^l \rho A\psi^2dx = \int_0^l \rho A \left(\frac{4}{l^2} x (l-x)\right)^2dx = \frac{8\rho Al}{15} \\
&k = \int_0^l EI \psi''^2 dx = \int_0^l EI \left(\frac{8}{l^2}\right)^2dx = \frac{64EI}{l^3} \\
&\omega = \sqrt{\frac{k}{m}} = \sqrt{\frac{64EI}{l^3}\frac{15}{8\rho Al}} = \frac{1}{l^2}\sqrt{120\frac{EI}{\rho A}} \approx 10.95\frac{1}{l^2}\sqrt{\frac{EI}{\rho A}}\,\si{rad/s}.
\end{align*}
The result obtained can be compared against the exact frequency, obtained by the same procedure applied to a fourth order polynomial satisfying both kinematic and dynamic boundary conditions. The analytical result is $\omega = \frac{1}{l^2}\sqrt{\frac{3024EI}{31\rho A}} \approx 9.86\frac{1}{l^2}\sqrt{\frac{EI}{\rho A}}\,\si{rad/s}$.

\parbox{.7\textwidth}{The cantilever can be analysed using a second order polynomial. In that case, the kinematic boundary conditions are $\psi(0) = \psi'(0) = 0$.} \hspace{1em}
\parbox{.25\textwidth}{\pictures{cantilever}}
\begin{align*}
&\psi(x) = \frac{1}{l^2} x^2 \\
&\psi''(x) = \frac{2}{l^2}.
\end{align*}
Again, the term $1/l^2$ is a scaling term providing consistency and does not modify the result of the calculations.
\begin{align*}
&m = \int_0^l \rho A\psi^2dx = \int_0^l \rho A \left(\frac{1}{l^2} x^2\right)^2dx = \frac{\rho Al}{5} \\
&k = \int_0^l EI \psi''^2 dx = \int_0^l EI \left(\frac{2}{l^2}\right)^2dx = \frac{4EI}{l^3} \\
&\omega = \sqrt{\frac{k}{m}} = \sqrt{\frac{4EI}{l^3}\frac{5}{\rho Al}} = \frac{1}{l^2}\sqrt{20\frac{EI}{\rho A}} \approx 4.47\frac{1}{l^2}\sqrt{\frac{EI}{\rho A}}\,\si{rad/s}.
\end{align*}
The exact frequency of the first mode of vibration of a cantilever is $\omega\approx3.52\frac{1}{l^2}\sqrt{\frac{EI}{\rho A}}\,\si{rad/s}$.

The last two structures can be analysed using a cubic polynomial, because the shape function must fulfill three kinematic boundary conditions.

For the application of the Rayleigh method with periodic shape functions, see the solution to Exercise \ref{moving_load}.
\end{Answer}
