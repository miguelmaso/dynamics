
\begin{Exercise}[label={wind_gust_impulse}]
The building of example 6 is hit by a sudden wind gust which results in the sudden application of horizontal forces distributed along the height of the building as shown in the picture. Assuming a vibration shape function that increases linearly with the height and neglecting damping, determine the maximum displacement at the top of the building.

\begin{center}
\pictures{buildingWind}
\end{center}

\shortAnswer $u_0 = 2.5cm$
\end{Exercise}



\begin{Answer}[ref={wind_gust_impulse}]
Now, the building from example \ref{helicopter_impulse} is exposed to sudden wind gust. In that case dynamics of the  structure are defined by a \emph{mass spring damper} system under a constant force suddenly applied.The maximum amplitude is governed by the stationary or homogeneous solution to the differential equation
\begin{align*}
m\ddot{u} + c + ku = F_0 \\
u(t) = \frac{F_0}{k}(1 -e^{\xi\omega t}\cos(\omega t))
\end{align*}

First of all, we need to determine the equivalent lateral force using the Rayleigh's method from the wind pressure $p_W$, using the same shape function $\psi$ from example 6,
\begin{align*}
F_0 = \int_0^h p_W\psi dz = \int_0^h \left(0.3 + (1-0.3)\frac{z}{h}\right)\frac{z}{h}dz = \left(\frac{0.3}{2} + \frac{0.7}{3}\right)h = \\
= \left(\frac{0.3}{2} + \frac{0.7}{3}\right)100 = 38.3KN
\end{align*}

The stiffness of the building can be estimated from the equivalent mass and the given period of the building
$$
k = \omega^2m = \left(\frac{2\pi}{T}\right)^2m = \left(\frac{2\pi}{5}\right)^2 2\cdot 10^6 = 3.125\cdot 10^6 N/m = 3125KN/m
$$

Finally, the maximum displacement is obtained from the equation of motion
$$
u_{max} = \frac{F_0}{k}(1+1) = \frac{38.3}{3125}2 = 0.025m = 2.5cm
$$
\end{Answer}
