\documentclass{problems}

% Loading shapes
% from https://openclipart.org/artist/dominiquechappard
% and https://openclipart.org/artist/mkhuda
\pgfdeclareimage[height=1.5cm]{calling}   {img/shapes/calling}
\pgfdeclareimage[height=1.5cm]{ballet}    {img/shapes/ballet}
\pgfdeclareimage[height=1.5cm]{dancing}   {img/shapes/dancing}
\pgfdeclareimage[height=1.5cm]{handwalk}  {img/shapes/handwalk}
\pgfdeclareimage[height=1.5cm]{happy}     {img/shapes/happy}
\pgfdeclareimage[height=1.5cm]{helicopter}{img/shapes/helicopter}
\pgfdeclareimage[height=1.5cm]{hip-hop}   {img/shapes/hip-hop}
\pgfdeclareimage[height=1.5cm]{hip-hop-2} {img/shapes/hip-hop-2}
\pgfdeclareimage[height=1.5cm]{jumping}   {img/shapes/jumping}
\pgfdeclareimage[height=1.5cm]{music-hall}{img/shapes/music-hall}
\pgfdeclareimage[height=1.5cm]{saxo}      {img/shapes/saxo}
\pgfdeclareimage[height=1.5cm]{stone}     {img/shapes/stone}
\pgfdeclareimage[height=1.5cm]{walking}   {img/shapes/walking}

% Macro for inserting pictures
\newcommand{\pictures}[1]{
    \begin{tikzpicture}
        \pic {#1};
    \end{tikzpicture}
}
\newcommand{\pictureslabel}[2]{
    \begin{tabular}[b]{c}
        \begin{tikzpicture}
            \pic {#1};
        \end{tikzpicture} \\
        (#2)
    \end{tabular}
}

% Predefined shapes
\tikzset{
    %
    % Supports
    %
    fixed/.pic = {
        \draw (-0.1,-.2) rectangle (.1,0);
    },
    pinned/.pic = {
        \draw (-.1,-.2) -- (.1,-.2) -- (0,0) -- cycle;
        \draw [fill=white] (0,0) circle (.05);
    },
    pinned0/.pic = {
        \draw (-.1,-.2) -- (.1,-.2) -- (0,0) -- cycle;
        \draw [fill=white] (0,-.05) circle (.05);
    },
    triangle/.pic = {
        \fill (-.1,-.2) -- (.1,-.2) -- (0,0) -- cycle;
    },
    %
    % Structures
    %
    basicPortico/.pic = {
        \coordinate (A) at (0.2,0);
        \coordinate (B) at (2.8,0);
        \draw [fill=lightgray] (0,2) rectangle (3,2.3);
        \draw [ultra thick] (A) -- +(0,2);
        \draw [ultra thick] (B) -- +(0,2);
    },
    basicBeam/.pic = {
        \coordinate (A) at (0,0);
        \coordinate (B) at (3,0);
        \draw [ultra thick] (A) -- (B);
    },
    fixedPortico/.pic = {
        \pic {basicPortico};
        \pic at (A) {fixed};
        \pic at (B) {fixed};
    },
    bracedPortico/.pic = {
        \pic {fixedPortico};
        \draw (A) -- (2.8,2);
        \draw (0.2,2) -- (B);
    },
    %
    % Example 1
    %
    portico/.pic = {
        \pic {fixedPortico};
        \path (1.5,2.3) node [above] {$m$};
        \path (A) -- +(0,2) node [midway,right] {$EI,h$};
    },
    portico3Pinned/.pic = {
        \pic {basicPortico};
        \coordinate (C) at (1.5,0);
        \draw [ultra thick] (C) -- +(0,2);
        \pic at (A) {pinned};
        \pic at (B) {pinned};
        \pic at (C) {pinned};
        \path (1.5,2.3) node [above] {$m$};
        \path (A) -- +(0,2) node [midway,right] {$EI,h$};
    },
    porticoBracing/.pic = {
        \pic {bracedPortico};
        \path (1.5,2.3) node [above] {$m$};
        \path (A) -- +(0,2) node [midway,right] {$EI,h$};
        \path (A) -- +(1.5,1.2) node [midway,below=4,rotate=35,inner sep=0] {$EA,l$};
        \path (B) node [left=10,inner sep=0] {$\theta$};
    },
    portico3PinnedSol/.pic = {
        \pic {portico3Pinned};
        \draw [dashed] (1,2) rectangle +(3,0.3);
        \draw [dashed] (A) to[out=40,in=270] +(1,2);
        \draw [dashed] (B) to[out=40,in=270] +(1,2);
        \draw [dashed] (C) to[out=40,in=270] +(1,2);
        \draw [<->] (3,2.6) -- +(1,0) node [midway,fill=white] {$u_0$};
        \draw [->] (4,2.15) -- +(1,0) node [above] {$F_0$};
    },
    porticoSol/.pic = {
        \pic {portico};
        \draw [dashed] (1,2) rectangle +(3,0.3);
        \draw [dashed] (A) to[out=90,in=270] +(1,2);
        \draw [dashed] (B) to[out=90,in=270] +(1,2);
        \draw [<->] (3,2.6) -- +(1,0) node [midway,fill=white] {$u_0$};
        \draw [->] (4,2.15) -- +(1,0) node [above] {$F_0$};
    },
    porticoBracingSol/.pic = {
        \pic {porticoBracing};
        \draw [dashed] (1,2) rectangle +(3,0.3);
        \draw [dashed] (A) to[out=90,in=270] +(1,2);
        \draw [dashed] (B) to[out=90,in=270] +(1,2);
        \draw [dashed] (A) -- +(3.6,2);
        \draw [dashed] (B) to[out=100,in=350] +(-1.6,2);
        \draw [<->] (3,2.6) -- +(1,0) node [midway,fill=white] {$u_0$};
        \draw [->] (4,2.15) -- +(1,0) node [above] {$F_0$};
    },
    columnPinned/.pic = {
        \coordinate (A) at (0,0);
        \coordinate (B) at (1,2);
        \draw [ultra thick] (A) to[out=40,in=270] (B);
        \draw [<->] (0,1.95) -- +(.95,0) node [midway,fill=white] {$u_0$};
        \draw [->] (A) -- +(-1,0) node [above] {$F_0$};
        \pic [rotate=180] at (B) {fixed};
    },
    columnFixed/.pic = {
        \coordinate (A) at (0,0);
        \coordinate (B) at (1,2);
        \draw [ultra thick] (A) to[out=90,in=270] (B);
        \draw [<->] (0,2) -- (.95,2) node [midway,fill=white] {$u_0$};
        \draw [->] (A) -- +(-1,0) node [above] {$F_0$};
        \draw [->] (B) -- +(1,0) node [above] {$F_0$};
        \draw [->] ([shift=(210:.5)]A) arc (210:330:.5) node [right] {$M_0$};
        \draw [->] ([shift=(30:.5)]B) arc (30:150:.5) node [midway,above] {$M_0$};
        \draw [ultra thin] (A) -- (60:.8);
        \draw ([shift=(55:.6)]A) node [right] {$\varphi_0$} arc (55:95:.6);
    },
    bracing/.pic = {
        \coordinate (A) at (0,0);
        \coordinate (B) at (3,2);
        \draw (A) -- (B);
        \pic at (A) {pinned};
        \draw [fill=black] (B) circle (.04);
        \draw [dashed] (A) -- (3.7,2) node (u0) {};
        \draw [->] ([shift=(0:.2)]B) -- (u0) node [midway,above] {$u_0$};
        \draw [->] ([shift=(0:.02)]u0) -- +(0:.7) node [right] {$F_0$};
        \draw [->] ([shift=(34:.02)]u0) -- +(34:.9) node [right] {$P$};
        \draw [<->] (25:3.8) -- +(25:.3) node [midway,below right] {$\delta$};
        \draw [->] ([shift=(0:.8)]A) node [right] {$\theta$} arc (0:34:.8);
    },
    %
    % Example 2
    %
    simpleBeam/.pic = {
        \pic {basicBeam};
        \pic at (A) {pinned};
        \pic at (B) {pinned};
        \path (A) -- (B) node [midway,above] {$EI,\rho,l,A$};
    },
    nSpanBeam/.pic = {
        \pic {basicBeam};
        \coordinate (C) at (2,0);
        \pic at (A) {pinned};
        \pic at (B) {pinned};
        \pic at (C) {pinned0};
        \path (A) -- (B) node [midway,above] {$EI,\rho,A$};
        \path (A) -- (C) node [midway,below,inner sep=1pt] {$l$};
        \path (C) -- (B) node [midway,below,inner sep=1pt] {$\alpha l$};
    },
    cantilever/.pic = {
        \pic {basicBeam};
        \pic [rotate=270] at (A) {fixed};
        \path (A) -- (B) node [midway,above] {$EI,\rho,l,A$};
    },
    fixedPinnedBeam/.pic = {
        \pic {basicBeam};
        \pic [rotate=270] at (A) {fixed};
        \pic at (B) {pinned};
        \path (A) -- (B) node [midway,above] {$EI,\rho,l,A$};
    },
    %
    % Example 3
    %
    porticoDimensions/.pic = {
        \pic {fixedPortico};
        \draw [<->] ([xshift=-10]A) -- +(0,2);
        \draw [<->] ([yshift=8]A) -- ([yshift=8]B);
        \path (A) -- +(0,2) node [midway,left=2,fill=white] {\SI{4}{m}};
        \path (A) -- (B) node [midway,above,fill=white] {\SI{5}{m}};
        \path (1.5,2.3) node [above] {\SI{100}{KN}};
    },
    porticoDimensions2/.pic = {
        \pic {bracedPortico};
        \draw (A) -- (2.8,2);
        \draw (0.2,2) -- (B);
        \pic at (A) {fixed};
        \pic at (B) {fixed};
    },
    porticoDimensions3/.pic = {
        \pic {bracedPortico};
        \draw (A) -- (2.8,2);
        \draw (0.2,2) -- (B);
        \draw [fill=lightgray] (0,2.3) rectangle (3,2.5);
        \pic at (A) {fixed};
        \pic at (B) {fixed};
        \path (1.5,2.5) node [above] {\SI{150}{KN}};
    },
    %
    % Example 4
    %
    bridge/.pic = {
        \coordinate (A) at (0,0);
        \coordinate (B) at (5,0);
        \path (A) -- (B) node [midway] (C) {};
        \path (C) node [below=10] (C0) {};
        \draw [dashed] (A) -- (B);
        \draw [ultra thick] (A) to[out=-20,in=200] (B);
        \pic at (A) {pinned};
        \pic at (B) {pinned};
        \draw [<-] (C) -- +(0,.5);
        \draw [<-] (C0) -- +(0,-.5);
        \path (C) node [above right] {\SI{5}{mm}};
    },
    %
    % Example 5
    %
    porticoForced/.pic = {
        \pic {porticoDimensions};
        \node [right] (C) at (B) {};
        \draw [<<->>] (C) -- +(1.5,0);
        \node [above right] at (C) {$a_g(t) = 2\sin(\Omega t)$};
    },
    %
    % Examples 6 and 7
    %
    building/.pic = {
        \draw [thick,gray,fill=lightgray] (0,0) rectangle (2,4);
        \draw (-1,0) -- (3,0);
        \foreach \x in {1,...,3} {
            \foreach \y in {1,...,5} {
                \draw [thick,gray,left color=blue!25, shading angle=45] (\x/2-.15, \y/1.5-.1) rectangle +(0.3,0.35);
            }
        }
    },
    buildingHelicopter/.pic = {
        \node at (0,4) {\pgfuseimage{helicopter}};
        \pic at (2,0) {building};
        \draw [dashed] (4,0) -- (5,4);
        \draw [dashed] (4,4) -- (5,4);
        },
    buildingWind/.pic = {
        \pic {building};
        \foreach \y in {0,...,8} {
            \draw [->] (-.5-\y/8, \y/2) -- (0, \y/2);
        }
        \coordinate (W0) at (-.5,0);
        \coordinate (W1) at (-1.5,4);
        \draw (W0) -- (W1);
        \node [above left] at (W0) {\SI{0.3}{KN/m}};
        \node [left] at (W1) {\SI{1}{KN/m}};
        \draw [dashed] (2,0) -- (3,4);
        \draw [dashed] (2,4) -- (3,4);
    },
    %
    % Example 8
    %
    masslessCable/.pic = {
        \draw (-.8,0) to[out=270, in=180] (-.5,-2);
        \draw (-.5,-2) to[out=0, in=270] (0,0) node [below right=3] {$l,EA$};
        \pic [rotate=180] {pinned};
        \draw [fill=lightgray] (-.8,0) circle (.2) node [left=5] {$m$};
    },
    %
    % Example 9
    %
    longBeam/.pic = {
        \coordinate (A) at (0,0);
        \coordinate (B) at (5,0);
        \draw [line width=2.5pt] (A) -- (B);
        \pic at (A) {pinned};
        \pic at (B) {pinned};
    },
    beamPointLoad/.pic = {
        \pic {longBeam};
        \draw [->,very thick] (1.5,1) node [left] {$P$} -- +(0,-.9);
        \draw [->] (1.6,.5) -- +(.6,0) node [above] {$v$};
    },
    %
    % Example 10
    %
    beamPeople/.pic = {
        \pic {longBeam};
        \node [above=-7] at (0.5,0) {\pgfuseimage{happy}};
        \node [above=-6] at (1.5,0) {\pgfuseimage{hip-hop}};
        \node [above=-5] at (2.5,0) {\pgfuseimage{handwalk}};
        \node [above=-5] at (3.5,0) {\pgfuseimage{dancing}};
        \node [above=-5] at (4.5,0) {\pgfuseimage{ballet}};
        \node [below] at (0,0) {\phantom{$T_f$}};
    },
    contactRatio/.pic = {
        \coordinate (O) at (0,0);
        \coordinate (X) at (5,0);
        \coordinate (Y) at (0,1.5);
        \coordinate (a) at (1.6,0);
        \coordinate (T) at (2.8,0);
        \coordinate (F) at (0,.4);
        \coordinate (sin) at (0.8,1.5);
        \draw [->] (O) -- (X) node [below right] {$t$};
        \draw [->] (O) -- (Y) node [left] {$F(t)$};
        \draw [thick] (O) .. controls +(sin) .. +(a) node [below] {$\alpha T_f$};
        \draw [thick] (T) node [below] {$T_f$} .. controls +(sin) .. +(a);
        \draw [dotted] (F) node [left] {$F_{avg}$} -- +(X);
    },
    storey2col/.pic = {
        \coordinate (A) at (0.2,0);
        \coordinate (B) at (2.8,0);
        \draw [fill=lightgray] (0,1) rectangle (3,1.2);
        \draw [ultra thick] (A) -- +(0,1);
        \draw [ultra thick] (B) -- +(0,1);
    },
    storey3col/.pic = {
        \coordinate (A) at (0.2,0);
        \coordinate (B) at (3.8,0);
        \coordinate (C) at (2,0);
        \draw [fill=lightgray] (0,1) rectangle (4,1.2);
        \draw [ultra thick] (A) -- +(0,1);
        \draw [ultra thick] (B) -- +(0,1);
        \draw [ultra thick] (C) -- +(0,1);
    },
    storeyPlanView/.pic = {
        \draw [fill=lightgray!30] (0,0) rectangle (2,2);
        \node[rectangle,draw,fill=black!80,inner sep=2pt] at (0.2,0.2) {};
        \node[rectangle,draw,fill=black!80,inner sep=2pt] at (1.8,0.2) {};
        \node[rectangle,draw,fill=black!80,inner sep=2pt] at (0.2,1.8) {};
        \node[rectangle,draw,fill=black!80,inner sep=2pt] at (1.8,1.8) {};
    },
    twoStorey/.pic = {
        \pic at (0,1.2) {storey2col};
        \pic at (0,0.0) {storey2col};
        \pic at (A) {fixed};
        \pic at (B) {fixed};
        \draw [<->] (A) ++(.4,1.2) -- ++(0,1) node [midway, right] {\SI{3}{m}};
        \draw [<->] (A) ++(.4,0.0) -- ++(0,1) node [midway, right] {\SI{3}{m}};
        \draw [-latex] (B) ++(.2,2.3) -- ++(.5,0) node [right] {$u_2$};
        \draw [-latex] (B) ++(.2,1.1) -- ++(.5,0) node [right] {$u_1$};
    },
    threeStorey/.pic = {
        \pic at (0,2.4) {storey3col};
        \pic at (0,1.2) {storey3col};
        \pic at (0,0.0) {storey3col};
        \pic at (A) {fixed};
        \pic at (B) {fixed};
        \pic at (C) {fixed};
        \draw [<->] (A) ++(.4,2.4) -- ++(0,1) node [midway, right] {\SI{3}{m}};
        \draw [<->] (A) ++(.4,1.2) -- ++(0,1) node [midway, right] {\SI{3}{m}};
        \draw [<->] (A) ++(.4,0.0) -- ++(0,1) node [midway, right] {\SI{3}{m}};
        \draw [-latex] (B) ++(.2,3.5) -- ++(.5,0) node [right] {$u_3$};
        \draw [-latex] (B) ++(.2,2.3) -- ++(.5,0) node [right] {$u_2$};
        \draw [-latex] (B) ++(.2,1.1) -- ++(.5,0) node [right] {$u_1$};
    },
    threeStoreyKM/.pic = {
        \pic {threeStorey};
        \draw (A) ++(-.5,3.6) node [left] {$m_3=\SI{e5}{kg}$};
        \draw (A) ++(-.5,2.4) node [left] {$m_2=\SI{1.5e5}{kg}$};
        \draw (A) ++(-.5,1.2) node [left] {$m_1=\SI{2e5}{kg}$};
        \draw (B) ++(.5,3.0) node [right] {$k_3=\SI{e7}{N/m}$};
        \draw (B) ++(.5,1.8) node [right] {$k_2=\SI{2e7}{N/m}$};
        \draw (B) ++(.5,0.6) node [right] {$k_1=\SI{3e7}{N/m}$};
    },
    %
    % Pneumatic hammer
    %
    concretePlate/.pic = {
        \def\cubex{5}
        \def\cubey{4}
        \def\cubez{.5}
        \def\fx{1}
        \def\fy{1}
        \def\fz{1}
        \draw [thin,fill=lightgray,semitransparent] (0,0,0) -- ++(-\cubex,0,0) -- ++(0,-\cubez,0) coordinate (a) -- ++(\cubex,0,0) coordinate (b) -- cycle;
        \draw [thin,fill=lightgray,semitransparent] (0,0,0) -- ++(0,0,-\cubey) -- ++(0,-\cubez,0) coordinate (c) -- ++(0,0,\cubey) -- cycle;
        \draw [thin,fill=lightgray,semitransparent] (0,0,0) -- ++(-\cubex,0,0) -- ++(0,0,-\cubey) -- ++(\cubex,0,0) -- cycle;
        \path (a) ++(0,-.6) coordinate (a1);
        \path (b) ++(.4,-.4) coordinate (b1);
        \path (c) ++(.4,-.4) coordinate (c1);
        \draw [<->] (a1) -- (a1 -| b) node [midway,above] {\SI{12}{m}};
        \draw [<->] (b1) -- (c1) node [midway,above,rotate=45] {\SI{12}{m}};
        \coordinate (F) at (-\cubex+\fx, \fx, -\cubey+\fy);
        \draw [thick,-latex] (F) -- ++(0,-\fz, 0);
        \draw [<->] (F) ++(0,.2,0) -- ++(-\fx,0,0) node [midway,below] {\SI{2}{m}};
        \draw [<->] (F) ++(0,.2,0) -- ++(0,0,-\fy) node [midway,below,rotate=45] {\SI{2}{m}};
    },
    %
    % Example NNN
    %
    singleTrain/.pic = {
        \tikzset{
            wheel/.style={draw, circle, thick, minimum width=8pt, inner sep=0pt, fill=gray!75},
            window/.style={draw, rectangle, very thin, left color=blue!20, shading angle=45, minimum width=7pt, minimum height=5pt, inner sep=0pt},
            wagon/.style={draw, very thin, shading angle=0}
        }
        \def\l{3}
        \def\h{.6}
        \path [draw]
            (0,1.2pt) coordinate (trolley) ++(-.5*\l,2pt) node (w1) [wheel] {}
            ++(\l,0) node (w2) [wheel] {};
        \path [wagon]
          [rounded corners=1pt] (w1.west |- trolley) -| (trolley -| .04-\l,0) |- ++(\l,\h) -| ++(\l-.08,-\h) 
          [sharp corners] -| (w2.east) -- (w2.west) |- (trolley) -| (w1.east) -- (w1.west) |- cycle;
        \foreach \pos in {0,.4,...,\l} {
            \path (trolley) ++(.9*\pos,.35) node [window] {};
            \path (trolley) ++(-.9*\pos,.35) node [window] {};
        }
    },
    fullTrain/.pic = {
        \pic at (0,0) {singleTrain};
        \def\l{3}
        \def\h{.7}
        \def\s{.04}
        \begin{scope}
            \clip (\l,-1.2pt) rectangle ++(\l,\h) ++(\s,-\h) rectangle ++(\s,\h) ++(\s,-\h) rectangle ++(\s,\h);
            \pic at (2*\l,0) {singleTrain};
        \end{scope}
        \begin{scope}
            \clip (-\l,-1.2pt) rectangle ++(-.1,\h) ++(-\s,-\h) rectangle ++(-\s,\h) ++(-\s,-\h) rectangle ++(-\s,\h);
            \pic at (-2*\l,0) {singleTrain};
        \end{scope}
    },
    trainOverBridge/.pic = {
        \tikzset{
            offset1/.style={yshift=.4pt},
            offset2/.style={yshift=.8pt}
        }
        \def\ya{-0.6pt}
        \def\yb{-1.2pt}
        \def\yc{-.9}
        \coordinate (annot) at (0, -.9);
        \pic at (0, 1.2pt) {fullTrain};
        \fill [lightgray,semitransparent] (-.4,0) coordinate (p1) rectangle ++(-2.7,-.5);
        \fill [lightgray,semitransparent] (2.6,0) coordinate (p2) rectangle ++(3.4,-.5);
        \draw [ultra thick,offset1] ([yshift=\ya]p1) -- ([yshift=\ya]p2);
        \path (p1) -- (p2) node [midway] (p3) {};
        \pic at ([yshift=\yb]p1) {triangle};
        \pic at ([yshift=\yb]p2) {triangle};
        \pic at ([yshift=\yb]p3) {triangle};
        \draw [<->] (p1 |- annot) -- (p3 |- annot) node [midway,above] {\SI{6}{m}};
        \draw [<->] (p2 |- annot) -- (p3 |- annot) node [midway,above] {\SI{6}{m}};
        \draw [<->] (-1.5, 1) -- ++(3,0) node [midway,above] {\SI{12}{m}};
    },
    rampForce/.pic = {
        \coordinate (O);
        \coordinate (t0) at (1,0);
        \coordinate (t1) at (5,0);
        \coordinate (P) at (0,1.4);
        \fill [lightgray,semitransparent] (O) -- (t0 |- P) -- (t1 |- P) -- (t1) -- cycle;
        \draw plot coordinates {(O) (t0 |- P) (t1 |- P)};
        \draw [->] (O) -- ++(6,0) coordinate (X) node [right] {$x$};
        \draw [->] (O) -- ++(0,2) coordinate (Y) node [left] {$y$};
        \draw [dotted] (t0) node [below] {$t_0$} -- (t0 |- P);
        \draw [dotted] (P) node [left] {$P$} -- (P -| t0);
    },
    infLongForce/.pic = {
        \pic {longBeam};
        \draw (-1,1) -- (2,1) coordinate (w) node [midway,above] {$w$};
        \draw [->] (w) ++(.3,0) -- ++(1,0) node [midway,above] {$v$};
        \foreach \x in {-.4,-.1,...,2} {
            \draw [-latex] (\x, 1) -- (\x, .1);
        }
    },
    elsaticSpectrum/.pic = {
        \begin{scope}[xscale=1.5,yscale=3]
        \fill [lightgray,semitransparent] (0,0) coordinate (O) -- (0,.2) coordinate (s0) -- (.2,.5) coordinate (s1) --
            (1,.5) coordinate (s2) -- plot [domain=1:2] ({\x}, {.5/\x}) coordinate (s3) -- plot [domain=2:4] ({\x}, {1/\x/\x}) -- (4,0) -- cycle;
        \draw plot coordinates {(s0) (s1) (s2)} plot [domain=1:2] ({\x}, {.5/\x}) plot [domain=2:4] ({\x}, {1/\x/\x});
        \draw [->] (O) -- (5,0) node [right] {$T (s)$};
        \draw [->] (O) -- (0,.8) node [left] {$S_a/g$};
        \node [left] at (s0) {$0.2$};
        \draw [dotted] (O |- s1) node [left] {$0.5$} -- (s1);
        \draw [dotted] (O -| s1) node [below] {$0.15$} -- (s1);
        \draw [dotted] (O -| s2) node [below] {$0.6$} -- (s2);
        \draw [dotted] (O -| s3) node [below] {$3.0$} -- (s3);
        \draw [<-] (s2) ++(.5,-.1) -- ++(.2,.2) node [above right] {$0.3/T$};
        \draw [<-] (s3) ++(.6,-.05) -- ++(.2,.2) node [above right] {$0.9/T^2$};
        \end{scope}
    },
    inelsaticSpectrum/.pic = {
        \begin{scope}[xscale=1,yscale=10]
        \fill [lightgray,semitransparent] (0,0) coordinate (O) -- (0,.2) coordinate (s0) -- (.2,.12) coordinate (s1) --
            (1,.12) coordinate (s2) -- plot [domain=1:2] ({\x}, {.12/\x}) coordinate (s3) -- plot [domain=2:4] ({\x}, {.24/\x/\x}) -- (4,0) -- cycle;
        \draw plot coordinates {(s0) (s1) (s2)} plot [domain=1:2] ({\x}, {.12/\x}) plot [domain=2:4] ({\x}, {.24/\x/\x});
        \draw [->] (O) -- (5,0) node [right] {$T (s)$};
        \draw [->] (O) -- (0,.3) node [left] {$S_a/g$};
        \node [left] at (s0) {$0.2$};
        \draw [dotted] (O |- s1) node [left] {$0.125$} -- (s1);
        \draw [dotted] (O -| s1) node [below] {$0.15$} -- (s1);
        \draw [dotted] (O -| s2) node [below] {$0.6$} -- (s2);
        \draw [dotted] (O -| s3) node [below] {$3.0$} -- (s3);
        \draw [<-] (s2) ++(.5,-.03) -- ++(.1,.1) node [above right] {$0.125(0.6/T)^{2/3}$};
        \draw [<-] (s3) ++(.8,-.02) -- ++(.08,.08) node [above right] {$0.125(0.2)^2(3/T)^{5/3}$};
        \end{scope}
    }
}





\title{Problems of dynamics of structures}
\author{J. Bonet, M. Masó, R. Chacón}
\entity{Escola Tècnica Superior d'Enginyeria de Camins, Canals i Ports de Barcelona}


\begin{document}

\maketitle

(For steel take $E=200GPa$ and for concrete $E=14GPa$)

\section{Free vibration of SDOF structures}


\example{1}
For the structures shown, determine the natural frequency of vibration using simple structural concepts.

\begin{center}
    \pictureslabel{portic3pinned}{a}
    \hspace{2em}
    \pictureslabel{portic}{b}
    \hspace{2em}
    \pictureslabel{porticbracing}{c}
\end{center}

Solutions: (a) $\omega = \sqrt{\frac{9EI}{mh^3}}$, (b) $\omega = \sqrt{\frac{24EI}{mh^3}}$ and (c) $\omega = \sqrt{\frac{24EI}{mh^3}+\frac{EA}{ml}\cos^2\theta}$.



\example{2}
For the structures shown, determine the natural frequency of vibration using Rayleigh's method.

\begin{center}
    \pictureslabel{simplebeam}{a}
    \pictureslabel{nspanbeam}{b} \\
    \pictureslabel{cantilever}{c}
    \pictureslabel{fixedpinnedbeam}{d}
\end{center}

Solutions: (a) $\omega \approx 10.95\frac{1}{l^2}\sqrt{\frac{EI}{\rho A}} rad/s$, (c) $\omega \approx 4.47\frac{1}{l^2}\sqrt{\frac{EI}{\rho A}} rad/s$



\example{3}
The portal frame structure shown has a weight of $100KN$. If the natural period of vibration is 0.9 seconds:
\begin{enumerate}[(a)]
    \item determine the lateral stiffness of the structure;
    \item determine the diameter of the steel cross-braces required to strenghten the structure by reducing the period to 0.3 seconds;
    \item determine the period if a further load of $50KN$ is added to the strenghtened structure.
\end{enumerate}

\begin{center}
    \pictureslabel{porticdimensions}{a}
    \pictureslabel{porticdimensions2}{b}
    \pictureslabel{porticdimensions3}{c}
\end{center}

Solutions: (a) $k = 487KN/m$, (b) $D = 1.4cm$, (c) $T = 0.37s$



\example{4}
In order to determine the dynamic properties of a simply supported bridge with a mass of $10^6Kg$, the midpoint is displaced $5mm$ by a jack and then suddenly released. At the end of 20 complete cycles, the time is 3 seconds and the peack displacement measured is $1mm$. Determine the natural period and damping ratio of the bridge.

\begin{center}
\pictures{bridge}
\end{center}

Solution: $T=0.15s$, $\xi=1.28\%$



\section{Forced vibration of SDOF structures}


\example{5} The portal frame of example 3 (a) is subject to a sinusoidal ground vibration with horizontal acceleration amplitude of $2m/s^2$. Assuming a damping ratio of $5\%$, determine the maximum displacement and maximum total acceleration of the frame when the period of floor vibration is: (a) 0.1 seconds; (b) 0.9 seconds and (c) 5 seconds.

\begin{center}
\pictures{porticforced}
\end{center}

Solutions: (a) $u_0 = 0.00064cm$, $\ddot{u}_0 = 0.011m\,s^{-1}$; (b) $u_0 = 41cm$, $\ddot{u}_0 = 20m\,s^{-1}$ and (c) $u_0 = 4.2cm$, $\ddot{u}_0 = 0.066m\,s^{-1}$



\example{6} A building has a height of $100m$, a square base measuring $20\times20m^2$, an average specific weight of $1\,500N/m^3$ and a natural period of vibration of 5 seconds. The top floor is hit by an helicopter with a mass of $10\,000Kg$ and travelling at $30m/s$. Determine the maximum deflection at the top assuming conservation of linear momentum and a vibration shape function that increases linearly with the height.

\begin{center}
\pictures{buildinghelicopter}
\end{center}

Solution: $u_0 = 12cm$



\example{7} The building of example 6 is hit by a sudden wind gust which results in the sudden application of horizontal forces distributed along the height of the building as shown in the picture. Assuming a vibration shape fuction that increases linearly with the height and neglecting damping, determine the maximum displacement at the top of the building.

\begin{center}
\pictures{buildingwind}
\end{center}



\example{8} A mass $m$ is released from a given height attached to a massless calbe of length $l$, area $A$ and Young's modulus $E$. If the cable is fixed at the point from which the mass is released, describe the motion/vibration of the mass. Determine the maximum stress in the cable and the lowest point reached by the mass.



\example{9} A point load $F=1KN$ moves along with constant speed $v=10m/s$ on a simply supported beam of length $l=1p\pi m$ as shown in the figure. The beam is made of concrete, has a rectanular section of $1m$ and an average density of $2\,800Kg/m^3$. Determine the deflection of the beam as function of time, the dynamic magnification factor and the maximum bending moment at the centre section.


%\section{Vibration of MDOF structures}


\newpage
\section{Solutions}


\example{1}
The natural frequency of a structure is obtained from the solution of the differential equation governing the displacement of a \emph{spring mass} system without excitation.
$$
m\ddot{u}+ku=0
$$
where $m$ is the mass of the idealized system and $k$ is the stiffness. The natural frequency depends on both constants, $\omega^2 = k/m$.

Every single structure can be decomposed in its elements and each element, analyzed by any of the standard methods. Here, to obtain the stiffness of each element, we impose a unit displacement $u_0$ generated by the corresponding force $F_0$. The stiffness of the structure is the sum of the stiffness of its components.

\begin{center}
    \pictures{portic3pinned_sol}
    \pictures{columnpinned}
\end{center}

The displacement of the columns can be analyized as a cantilever using static analysis concepts:
$$
u_0 = \frac{F_0h^3}{3EI} \quad \rightarrow \quad k_{column} = \frac{3EI}{h^3}
$$
Finally, the stiffness and the frqeuncy of the structure are
$$
k = 3k_{column} = \frac{9EI}{h^3} \quad , \quad \omega = \sqrt{\frac{9EI}{mh^3}}
$$

\begin{center}
    \pictures{portic_sol}
    \pictures{columnfixed}
\end{center}

Analogously, the second structure can be analyzed combining the stiffness of the colunms. In that case, rotation $\varphi_0 = u_0/h$ generated by the moment reaction $M_0$ has been imposed to the quivalent beams. The moment reaction must satisfy global equilibrium:
$$
\sum M = 2M_0 -F_0h = 0
$$
And from static analysis, the rotation generated by the moment is
$$
\varphi_0 = \frac{M_0h}{6EI}
$$
Substituting the moment and the rotation into the above expression gives
$$
\frac{u_0}{h} = \frac{F_0h}{12EI} \quad \rightarrow \quad k_{column} = \frac{12EI}{h^3}
$$
The lateral stiffness and frequency of the structure are
$$
k = 2k_{column} = \frac{24EI}{h^3} \quad , \quad \omega = \sqrt{\frac{24EI}{mh^3}}
$$

\begin{center}
    \pictures{porticbracing_sol}
    \pictures{bracing}
\end{center}

The last structure adds two braces and its stiffness shall be added, but only one of them is contributing, since the bracing under compression buckles. The stifness of a brace is
$$
\delta = \frac{Pl}{EA} \quad \rightarrow \quad k_{brace} = \frac{EA}{l}\cos^2\theta
$$
and the lateral stiffness and frequency of the structure are
$$
k = 2k_{column} + k_{brace} = \frac{24EI}{h^3}+\frac{EA}{l}\cos^2\theta \quad , \quad 
\omega = \sqrt{\frac{24EI}{mh^3}+\frac{EA}{ml}\cos^2\theta}
$$



\example{2}
The Rayleigh's method is based on assuming the vibration to be given in terms of a pre-determined shape function $\psi$ as $u(x,t) = u_0(t)\psi(x)$. This assumption allows to find an equivalent mass and stiffness associated to that mode of vibration and thus, the frequency.

\parbox{.7\textwidth}{For the simple beam, a possible shape function would be a parabola satisfying the kynematic boundary conditions, $\psi(0) = \psi(l) = 0$.} \hspace{1em}
\parbox{.25\textwidth}{\pictures{simplebeam}}
\begin{align*}
&\psi(x) = \frac{4}{l^2} x (l-x) \\
&\psi''(x) = \frac{8}{l^2}
\end{align*}
where the term $4/l^2$ is a scaling parameter and does not modify the solution. The equivalent mass and stiffness and frequency are
\begin{align*}
&m = \int_0^l \rho A\psi^2dx = \int_0^l \rho A \left(\frac{4}{l^2} x (l-x)\right)^2dx = \frac{8\rho Al}{15} \\
&k = \int_0^l EI \psi''^2 dx = \int_0^l EI \left(\frac{8}{l^2}\right)^2dx = \frac{64EI}{l^3} \\
&\omega = \sqrt{\frac{k}{m}} = \sqrt{\frac{64EI}{l^3}\frac{15}{8\rho Al}} = \frac{1}{l^2}\sqrt{120\frac{EI}{\rho A}} \approx 10.95\frac{1}{l^2}\sqrt{\frac{EI}{\rho A}} rad/s
\end{align*}
The obtained result can be compared against the exact frequency, obtained by the same procedure applied to a fourth order polynomial satisfying both kynematic and dynamic boundary conditions. The analytical result is $\omega = \frac{1}{l^2}\sqrt{\frac{3024EI}{31\rho A}} \approx 9.86\frac{1}{l^2}\sqrt{\frac{EI}{\rho A}} rad/s$.

\parbox{.7\textwidth}{The cantilever can be analyzed using a second order polynomial. In that case, the kynematic boundary conditions are $\psi(0) = \psi'(0) = 0$.} \hspace{1em}
\parbox{.25\textwidth}{\pictures{cantilever}}
\begin{align*}
&\psi(x) = \frac{1}{l^2} x^2 \\
&\psi''(x) = \frac{2}{l^2}
\end{align*}
Again, the term $1/l^2$ is a scaling term providing consistency and does not modify the result of the calculations.
\begin{align*}
&m = \int_0^l \rho A\psi^2dx = \int_0^l \rho A \left(\frac{1}{l^2} x^2\right)^2dx = \frac{\rho Al}{5} \\
&k = \int_0^l EI \psi''^2 dx = \int_0^l EI \left(\frac{2}{l^2}\right)^2dx = \frac{4EI}{l^3} \\
&\omega = \sqrt{\frac{k}{m}} = \sqrt{\frac{4EI}{l^3}\frac{5}{\rho Al}} = \frac{1}{l^2}\sqrt{20\frac{EI}{\rho A}} \approx 4.47\frac{1}{l^2}\sqrt{\frac{EI}{\rho A}} rad/s
\end{align*}
And the exact frquency of the first mode of vibration is $\omega\approx3.52\frac{1}{l^2}\sqrt{\frac{EI}{\rho A}}rad/s$.

The last two strutures can be analyzed using a cubic polynomial, since the shape function must fullfill three kynematic boundary conditions.

For the application of Rayleigh's method with periodic shape functions, see the solution to example 9.



\example{3}
Given that the natural period of vibration of the portic is $T=0.9$ seconds, the frequency is computed as
$$
\omega = \frac{2\pi}{T} = \frac{2\pi}{0.9} = 6.98rad/s
$$
Then, the lateral stiffness is computed from the frequency and the mass of the structure,
$$
\omega^2 = \frac{k}{m} \quad \rightarrow \quad
k = \omega^2m = \omega^2\frac{P}{g} = 6.98^2\frac{100}{10} = 487KN/m
$$

The goal of the second step is to determine the diameter of the steel cross-braces required to strenghten the structure by reducing the period to 0.3 seconds. First of all, the new stiffness is obtained following the same procedure,
\begin{align*}
\omega = \frac{2\pi}{T} = \frac{2\pi}{0.3} = 20.94rad/s \\
k = \omega^2\frac{P}{g} = 20.94^2\frac{100}{10} = 4384KN/m
\end{align*}
The bracing system should provide the additional stiffness,
$$
k = k_{portic} + k_{bracing} \quad \rightarrow \quad k_{bracing} = 4384 - 487 = 3897KN/m
$$

We will consider the stiffness of one brace because the one under compression can buckle. 
Following the solution of example 1 (c), the area can be obtained from
\begin{align*}
k_{brace} &= \frac{EA}{l}\cos^2\theta \quad \rightarrow \quad A = \frac{k_{brace}l}{E\cos^2\theta} \\
l &= \sqrt{4^2 + 5^2} = 6.4m \\
\cos\theta &= \frac{5}{6.4} = 0.78 \\
A &= \frac{3897\cdot 6.4}{2\cdot 10^8\cdot 0.78} = 1.6\cdot 10^{-4} m^2 = 1.6cm^2
\end{align*}
and the diameter is obtained from the area,
$$
D = 2\sqrt{\frac{A}{\pi}} = 2\sqrt{\frac{1.6}{\pi}} = 1.4cm
$$

The last part of the example consits on computing the new period of vibration if a further load of $50KN$ is added to the strenghtened structure. Using the new lateral stiffness, the frequency is
$$
\omega = \sqrt{\frac{k}{m}} = \sqrt{\frac{4384}{15}} = 17.1 rad/s
$$
and the new period is
$$
T = \frac{2\pi}{\omega} = \frac{2\pi}{17.1} = 0.37s
$$



\example{4}
The solution to the differential equation governing free vibration of a \emph{mass spring damper} system is governed by the natural and damped frequencies and the damping ratio:
$$
u(t) = e^{-\xi\omega t}\sin(\omega_D t) \quad ; \quad \omega_D = \omega\sqrt{1-\xi^2}
$$
For simplicity, we will assume a small damping and $\sqrt{1-\xi^2}\rightarrow1$, thus, the natural period can be obtained directly from the measured values
$$
T \approx T_D = \frac{3}{20} = 0.15s
$$
And the damping ratio $\xi$ is directly related to the decay coefficient determined by the relation among two consecutive oscillations from a damped period
$$
\frac{u(t)}{u(t+T_D)} = e^{\xi\omega T_D} = e^{\frac{2\pi\xi}{\sqrt{1-\xi^2}}} \quad \rightarrow \quad
\frac{u_0}{u_1} \frac{u_1}{u_2} \dots \frac{u_{n-1}}{u_n} = \frac{u_0}{u_n} = e^{\frac{2n\pi\xi}{\sqrt{1-\xi^2}}}
$$
Since $\xi$ is small, the damping ratio can be found as
\begin{align*}
2n\pi\xi = \log\left(\frac{u_0}{u_n}\right) \quad \rightarrow \quad
\xi = \frac{1}{2n\pi} \log\left(\frac{u_0}{u_n}\right) = \frac{1}{2\cdot20\pi} \log\left(\frac{5}{1}\right) = 0.0128 \\
\xi = 1.28\%
\end{align*}



\example{5}
The maximum displacements and accelerations of a vibrating structure under sinusoidal excitation are obtained from the stationary or homogeneous solution to the differential equation,
$$
u(t) = \frac{F_0}{k}H\sin(\Omega t + \varphi - \Delta\varphi)
$$
where $H$ is known as the magnification factor and is equal to
$$
H = \frac{1}{\sqrt{(1-\gamma^2)^2 + 4\xi^2\gamma^2}} \quad ; \quad
\gamma = \frac{\Omega}{\omega} = \frac{T_{struct}}{T_{force}}
$$

From the solution to example 3 (a) we know that the period is $T=0.9$ seconds and the frequency is $\omega = 2\pi/T \approx 7rad/s$ and the lateral stiffness is $k=487KN/m$. The damping ratio is $\xi=5\%$.

The amplitude $F_0$ of the external force is computed from the amplitude of the ground acceleration and the moving mass of the structure
$$
F_0 = ma_{0} = \frac{Pa_0}{g} = \frac{100\cdot2}{10} = 20KN
$$

Finally, the maximum displacements will depend on the period $\Omega$ of the external force,
\begin{align*}
&u_0 = \frac{F_0}{k}H = \frac{20}{487}H = 0.041H\ m = 4.1H\ cm \\
&\ddot{u}_0 = \frac{F_0}{m}\gamma^2H = \Omega^2u_0
\end{align*}


When the period of the ground acceleration is $T_g=0.15s$, the maximum displacements and accelerations are
\begin{align*}
&\gamma = \frac{0.9}{0.15} = 9 \quad ; \quad
H = \frac{1}{\sqrt{(1-9^2)^2 + 4\cdot 0.05^2\cdot 9^2}} = \frac{1}{\sqrt{80^4 + \cdots}} = 0.00015 \\
&u_0 = 4.1\cdot 0.00015 = 0.00064 cm \\
&\ddot{u}_0 = \left(\frac{2\pi}{0.15}\right)^2 \cdot 0.00064 = 1.1cm\,s^{-2} = 0.011m\,s^{-2}
\end{align*}
This situation is \emph{mass dominated}, also known as \emph{vibration isolation}.

When the period of the ground acceleration is $T_g=0.9s$, the maximum values are
\begin{align*}
&\gamma = \frac{0.9}{0.9} = 1 \quad ; \quad
H = \frac{1}{\sqrt{(1-1^2)^2 + 4\cdot 0.05^2\cdot 1^2}} = \frac{1}{\sqrt{0 + 4\cdot 0.05^2}} = 10 \\
&u_0 = 4.1\cdot 10 = 41 cm \\
&\ddot{u}_0 = \left(\frac{2\pi}{0.9}\right)^2 \cdot 41 = 2000cm\,s^{-2} = 20m\,s^{-2}
\end{align*}
The \emph{ressonance} situation is \emph{damping dominated}.

And for a period $T_g=5s$, the maximum values are
\begin{align*}
&\gamma = \frac{0.9}{5} = 0.18 \quad ; \quad
H = \frac{1}{\sqrt{(1-0.18^2)^2 + 4\cdot 0.05^2\cdot 0.18^2}} = 1.033 \\
&u_0 = 4.1\cdot 1.033 = 4.2 cm \\
&\ddot{u}_0 = \left(\frac{2\pi}{5}\right)^2 \cdot 4.2 = 6.6cm\,s^{-2} = 0.066m\,s^{-2}
\end{align*}
Which is a practically \emph{static} situation or \emph{stiffness dominated}.

\example{6}
When the building is hit by the helicopter, the impulse or linear momentum is preserved. To compute the linear momentum of the building, we need to know the equivalent mass and we can use the Rayleigh's method for that purpose. As it is suggested, we will use a linear shape function
\begin{align*}
&u = u_0\psi \quad ; \quad \psi = \frac{z}{h} \\
&m = \int_0^h \rho A\psi^2dz = \rho A\int_0^h\frac{z^2}{h^2}dz = \frac{1}{3}\rho Ah = \frac{1}{3} 1500 \cdot 400 \cdot 100 = \\
& \pushright{= 2\cdot 10^7N = 2\cdot 10^6Kg}
\end{align*}

The impulse of the helicopter is
$$
I = (mv)_{helicopter} = 10^4 \cdot 30 = 3\cdot 10^5 Kg\,m\,s^-1
$$
which is transferred to the building. Then, the maximum deflection at the top of the building is inferred
\begin{align*}
&\dot{u}_0 = \frac{I}{m} \quad \rightarrow \quad u_0 = \frac{I}{m\omega} \\
&\omega = \frac{2\pi}{T} = \frac{2\pi}{5} = 1.25rad\,s^{-1} \\
&u_0 = \frac{3\cdot 10^5}{2\cdot 10^6\cdot 1.25} = 0.12m = 12cm
\end{align*}




\end{document}
